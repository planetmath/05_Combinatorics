\documentclass[12pt]{article}
\usepackage{pmmeta}
\pmcanonicalname{TaitColoring}
\pmcreated{2013-03-22 15:10:29}
\pmmodified{2013-03-22 15:10:29}
\pmowner{marijke}{8873}
\pmmodifier{marijke}{8873}
\pmtitle{Tait coloring}
\pmrecord{9}{36927}
\pmprivacy{1}
\pmauthor{marijke}{8873}
\pmtype{Definition}
\pmcomment{trigger rebuild}
\pmclassification{msc}{05C15}
\pmrelated{ColoringsOfPlaneGraphs}
\pmrelated{VizingsTheorem}

\usepackage{amssymb}
\begin{document}
\PMlinkescapeword{classes}
\PMlinkescapeword{terms}
\PMlinkescapeword{classes}
\PMlinkescapeword{proportion} 

A {\bf Tait coloring} of a \PMlinkname{trivalent}{Valency} (aka cubic) graph is a coloring of its edges with only three colors, such that at each vertex the colors of the three edges there are different. After Peter~G.\ Tait (1831--1901), Scottish physicist, who in 1880 proved the following

{\bf Theorem (Tait)}$\quad$ A bridge-free trivalent plane graph can be
face-colored with 4 colors if and only if it can be edge-colored with 3
colors.

This \PMlinkid{introduction to face- and edge-colorings of plane graphs}{6925}
has a {\bf proof of the theorem}.

To put this in a modern context, {\bf Vizing's theorem} says that graphs $G$
fall into two classes: those that can be colored with $\Delta(G)$ colors and
those that need $\Delta(G)+1$ colors, where $\Delta(G)$ is the largest valency
that occurs in $G$. When applied to trivalent graphs that means 3 and 4
colors, respectively. It is known that ``almost all'' are in the first class
--- this expression has a technical meaning (for increasing graph size, the
proportion of them in the second class tends to zero).

Thanks to Tait's result, it is a corollary of the {\bf four-color theorem}
that all planar trivalent graphs are in the first class (can be edge-3-colored).
The converse is not the case. Many trivalent graphs, in fact almost all of
them, can be edge-3-colored and yet are not planar. $K_{3,3}$ is one example.

% In terms of a Venn diagram: among the trivalent graphs,
% planar and second class ones are disjoint.
%
% \begin{center}
% \begin{picture}(400,100)(-200,-50)
% \put(- 80,+36){\Huge C\,L\,A\,S\,S \ \,I}
% \put(- 80,+14){\LARGE needs 3 colors}
% \put(+ 61, +3){\bf C\,L\,A\,S\,S \ \,I\,I}
% \put(+ 61,-10){\small\bf needs 4 colors}
% \put(-100,0){\oval(100,40)}
% \put(+100,0){\oval(100,40)}
% \put(-136,-4){\textbf{\textit{P\,L\,A\,N\,A\,R}}}
% \put(-50,-30){\LARGE\it N O N - P L A N A R}
% \end{picture}
% \end{center}

The Petersen graph is an example of a trivalent graph that needs 4 colors.
%%%%%
%%%%%
\end{document}
