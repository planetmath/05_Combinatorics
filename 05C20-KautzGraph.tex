\documentclass[12pt]{article}
\usepackage{pmmeta}
\pmcanonicalname{KautzGraph}
\pmcreated{2013-03-22 16:22:55}
\pmmodified{2013-03-22 16:22:55}
\pmowner{wati}{15191}
\pmmodifier{wati}{15191}
\pmtitle{Kautz graph}
\pmrecord{7}{38526}
\pmprivacy{1}
\pmauthor{wati}{15191}
\pmtype{Definition}
\pmcomment{trigger rebuild}
\pmclassification{msc}{05C20}
%\pmkeywords{Graph}
%\pmkeywords{digraph}
\pmrelated{DeBruijnDigraph}

% this is the default PlanetMath preamble.  as your knowledge
% of TeX increases, you will probably want to edit this, but
% it should be fine as is for beginners.

% almost certainly you want these
\usepackage{amssymb}
\usepackage{amsmath}
\usepackage{amsfonts}

% used for TeXing text within eps files
%\usepackage{psfrag}
% need this for including graphics (\includegraphics)
\usepackage{graphicx}
% for neatly defining theorems and propositions
%\usepackage{amsthm}
% making logically defined graphics
%%%\usepackage{xypic}

% there are many more packages, add them here as you need % them
%\usepackage{epsfig}
% define commands here

\begin{document}
\noindent
The {\em Kautz graph} ${\cal K}_K^{N + 1}$ is a digraph (directed
graph) of degree $K$ and dimension $N+ 1$, which has $(K +1)K^{N}$
vertices labeled by all possible strings $s_0 \cdots s_N$ of length $N
+ 1$ which are composed of characters $s_i$ chosen from an alphabet
$A$ containing $K + 1$ distinct symbols, subject to the condition that
adjacent characters in the string cannot be equal ($s_i \neq s_{i+
1}$).
\vspace*{0.1in}

\noindent
The Kautz graph  ${\cal K}_K^{N + 1}$ has $(K + 1)K^{N + 1}$ edges
\begin{equation}
\{(s_0 s_1 \cdots s_N,
s_1 s_2 \cdots s_N s_{N + 1})| \; s_i \in A \; s_i \neq s_{i
  + 1} \} \,.
\label{eq:}
\end{equation}
It is natural to label each such edge of  ${\cal K}_K^{N + 1}$
as $s_0s_1 \cdots s_{N + 1}$, giving a one-to-one correspondence
between edges of    the Kautz graph  ${\cal K}_K^{N + 1}$
and vertices of the Kautz graph
${\cal K}_K^{N + 2}$.
\vspace*{0.1in}

\noindent
{\bf Example:}

\noindent
The Kautz graph   ${\cal K}_2^{2}$ has 6 nodes, and is depicted in the
following figure (using the alphabet $A$ =\{0, 1, 2\})

\begin{center}
%\epsfig{file=c:wati22.eps,width=10cm}
\includegraphics{C:wati22-8.eps}
\end{center}

\vspace*{0.1in}

\noindent
{\bf Properties:}

\noindent $\bullet$
The diameter of the Kautz graph
${\cal K}_K^{N}$ is $N$ \\
(For example, there is a path of length $N$ from $x$ to $y$
achieved by the
sequence of edges $(x_0 x_1 \cdots x_N, x_1 \cdots x_N y_0)$, \ldots
$(x_N y_0 \cdots y_{N -1}, y_0 \cdots y_N)$ unless $x_N = y_0$ in
which case there is a similar path of length $N -1$ beginning
$(x_0 x_1 \cdots x_{N-1} y_0, x_1 \cdots x_{N-1} y_0 y_1), \ldots$.)

\noindent $\bullet$ 
Kautz graphs are closely related to {\em de Bruijn graphs}, which are
defined similarly but without the condition $s_i \neq s_{i + 1}$, and
with an alphabet of only $K$ symbols for the degree $K$ de Bruijn graph.

\noindent $\bullet$
For a fixed degree $K$ and number of vertices $V = (K + 1) K^N$, the
Kautz graph has the smallest diameter of any possible directed graph
with $V$ vertices and degree $K$.

\noindent $\bullet$ All Kautz graphs have Eulerian cycles\\ (An Eulerian cycle
is one which visits each edge exactly once-- This result follows
because Kautz graphs have in-degree equal to out-degree for each
node)

\noindent $\bullet$ All Kautz graphs have a Hamiltonian cycle\\
(This result follows from the correspondence described above
between edges of    the Kautz graph  ${\cal K}_K^{N}$
and vertices of the Kautz graph
${\cal K}_K^{N + 1}$; a Hamiltonian cycle on ${\cal K}_K^{N + 1}$ is
given by an Eulerian cycle on ${\cal K}_K^{N}$)

\noindent $\bullet$ A degree-$k$ Kautz graph has $k$ disjoint
paths from any node  $x$
to any other node $y$.


%%%%%
%%%%%
\end{document}
