\documentclass[12pt]{article}
\usepackage{pmmeta}
\pmcanonicalname{HallsMarriageTheoremProofOf}
\pmcreated{2013-03-22 12:45:12}
\pmmodified{2013-03-22 12:45:12}
\pmowner{mps}{409}
\pmmodifier{mps}{409}
\pmtitle{Hall's marriage theorem, proof of}
\pmrecord{10}{33059}
\pmprivacy{1}
\pmauthor{mps}{409}
\pmtype{Proof}
\pmcomment{trigger rebuild}
\pmclassification{msc}{05D15}

% this is the default PlanetMath preamble.  as your knowledge
% of TeX increases, you will probably want to edit this, but
% it should be fine as is for beginners.

% almost certainly you want these
\usepackage{amssymb}
\usepackage{amsmath}
\usepackage{amsfonts}

% used for TeXing text within eps files
%\usepackage{psfrag}
% need this for including graphics (\includegraphics)
%\usepackage{graphicx}
% for neatly defining theorems and propositions
%\usepackage{amsthm}
% making logically defined graphics
%%%\usepackage{xypic}

% there are many more packages, add them here as you need them

% define commands here

\newcommand{\Prob}[2]{\mathbb{P}_{#1}\left\{#2\right\}}
\newcommand{\norm}[1]{\left\|#1\right\|}

% Some sets
\newcommand{\Nats}{\mathbb{N}}
\newcommand{\Ints}{\mathbb{Z}}
\newcommand{\Reals}{\mathbb{R}}
\newcommand{\Complex}{\mathbb{C}}

\newcommand{\size}[1]{\left|#1\right|}
\begin{document}
We prove Hall's marriage theorem by induction on $\size{S}$, the size of $S$.  

The theorem is trivially true for $|S|=0$.

Assuming the theorem true for all $\size{S}<n$, we prove it for $\size{S}=n$.

\newcommand{\x}[1]{#1\setminus\{x\}}
First suppose that we have the stronger condition 
$$
\size{\cup T} \ge \size{T}+1
$$
for all $\emptyset \ne T \subset S$.  Pick any $x\in S_n$ as the representative of $S_n$; we must choose an SDR from
$$
S' = \left\{\x{S_1},\ldots,\x{S_{n-1}}\right\}.
$$
But if
$$
\{\x{S_{j_1}},...,\x{S_{j_k}}\} = T'\subseteq S'
$$
then, by our assumption,
$$
\size{\cup T'} \ge \size{\cup_{i=1}^{k} S_{j_i}}-1 \ge k.
$$
By the already-proven case of the theorem for $S'$ we see that we can indeed pick an SDR for $S'$.

Otherwise, for some
$\emptyset\ne T \subset S$ we have the ``exact'' size
$$
\size{\cup T}=\size{T}.
$$
Inside $T$ itself, for any $T'\subseteq T\subset S$ we have
$$
\size{\cup T'}\ge\size{T'},
$$
so by an already-proven case of the theorem we can pick an SDR for $T$.

It remains to pick an SDR for $S\setminus T$ which avoids all elements of $\cup T$ (these elements are in the SDR for $T$).
To use the already-proven case of the theorem (again) and do this, we must show that for any $T'\subseteq S\setminus T$, even after discarding elements of $\cup T$ there remain enough elements in $\cup T'$: we must prove
$$
\size{\cup T' \setminus \cup T} \ge \size{T'}.
$$

But
\begin{eqnarray}
\size{\cup T' \setminus \cup T} &= \size{\bigcup(T\cup T')} -
\size{\cup T} \ge\\
&\ge \size{T\cup T'} - \size{T} =\\
&= \size{T} + \size{T'} - \size{T} = \size{T'},
\end{eqnarray}
using the disjointness of $T$ and $T'$.  So by an already-proven case of the theorem, $S\setminus T$ does indeed have an SDR which avoids all elements of $\cup T$.

This \PMlinkescapetext{completes} the proof.
%%%%%
%%%%%
\end{document}
