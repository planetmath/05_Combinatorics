\documentclass[12pt]{article}
\usepackage{pmmeta}
\pmcanonicalname{Clique}
\pmcreated{2013-03-22 12:30:53}
\pmmodified{2013-03-22 12:30:53}
\pmowner{Mathprof}{13753}
\pmmodifier{Mathprof}{13753}
\pmtitle{clique}
\pmrecord{13}{32752}
\pmprivacy{1}
\pmauthor{Mathprof}{13753}
\pmtype{Definition}
\pmcomment{trigger rebuild}
\pmclassification{msc}{05C69}
%\pmkeywords{subgraph}
%\pmkeywords{maximal complete subgraph}
\pmrelated{IndependentSetAndIndependenceNumber}
\pmdefines{clique number}
\pmdefines{maximum clique}

\endmetadata

\usepackage{amssymb}
\usepackage{amsmath}
\usepackage{amsfonts}
\begin{document}
\PMlinkescapeword{maximal}A maximal \PMlinkid{complete}{1757} subgraph of a graph is a \emph{clique}, and the \emph{clique number} $\omega(G)$ of a graph $G$ is the \PMlinkescapephrase{maximal order} maximal order of a clique in $G$. Simply, $\omega(G)$ is the maximal order of a \PMlinkescapetext{complete} subgraph of $G$. Some authors however define a clique as any \PMlinkescapetext{complete} subgraph of $G$ and refer to the other definition as \textit{maximum clique}.


\footnotesize{Adapted with permission of the author from \emph{\PMlinkescapetext{Modern Graph Theory}} by B\'{e}la Bollob\'{a}s, published by Springer-Verlag New York, Inc., 1998.}
%%%%%
%%%%%
\end{document}
