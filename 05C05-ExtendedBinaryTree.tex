\documentclass[12pt]{article}
\usepackage{pmmeta}
\pmcanonicalname{ExtendedBinaryTree}
\pmcreated{2013-03-22 12:31:36}
\pmmodified{2013-03-22 12:31:36}
\pmowner{aoh45}{5079}
\pmmodifier{aoh45}{5079}
\pmtitle{extended binary tree}
\pmrecord{8}{32766}
\pmprivacy{1}
\pmauthor{aoh45}{5079}
\pmtype{Data Structure}
\pmcomment{trigger rebuild}
\pmclassification{msc}{05C05}
\pmrelated{BinaryTree}
\pmrelated{CompleteBinaryTree}
\pmrelated{ExternalPathLength}
\pmrelated{WeightedPathLength}
\pmrelated{MinimumWeightedPathLength}
\pmdefines{external node}
\pmdefines{internal node}

\endmetadata

\usepackage{amssymb}
\usepackage{amsmath}
\usepackage{amsfonts}
\usepackage{graphicx}
\begin{document}
An \emph{extended binary tree} is a transformation of any binary tree into a complete binary tree.  This transformation consists of replacing every null subtree of the original tree with ``special nodes.''  The nodes from the original tree are then \emph{internal nodes}, while the ``special nodes'' are \emph{external nodes}.

For instance, consider the following binary tree.

\begin{center}
\includegraphics{tree1}
\end{center}

The following tree is its extended binary tree.  Empty circles represent internal nodes, and filled circles represent external nodes.

\begin{center}
\includegraphics{tree2}
\end{center}

Every internal node in the extended tree has exactly two children, and every external node is a leaf.  The result is a complete binary tree.
%%%%%
%%%%%
\end{document}
