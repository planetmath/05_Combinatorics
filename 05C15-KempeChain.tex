\documentclass[12pt]{article}
\usepackage{pmmeta}
\pmcanonicalname{KempeChain}
\pmcreated{2013-05-16 21:17:19}
\pmmodified{2013-05-16 21:17:19}
\pmowner{marijke}{8873}
\pmmodifier{unlord}{1}
\pmtitle{Kempe chain}
\pmrecord{7}{36934}
\pmprivacy{1}
\pmauthor{marijke}{1}
\pmtype{Definition}
\pmcomment{trigger rebuild}
\pmclassification{msc}{05C15}
\pmrelated{ColoringsOfPlaneGraphs}
\pmrelated{ProofOfVizingsTheoremForGraphs}

\usepackage{amssymb}
% \usepackage{amsmath}
% \usepackage{amsfonts}

% used for TeXing text within eps files
%\usepackage{psfrag}
% need this for including graphics (\includegraphics)
%\usepackage{graphicx}

% for neatly defining theorems and propositions
%\usepackage{amsthm}
% making logically defined graphics
%%%\usepackage{xypic}

% there are many more packages, add them here as you need them

% define commands here %%%%%%%%%%%%%%%%%%%%%%%%%%%%%%%%%%
% portions from
% makra.sty 1989-2005 by Marijke van Gans %
                                          %          ^ ^
\catcode`\@=11                            %          o o
                                          %         ->*<-
                                          %           ~
%%%% CHARS %%%%%%%%%%%%%%%%%%%%%%%%%%%%%%%%%%%%%%%%%%%%%%

                        %    code    char  frees  for

\let\Para\S             %    \Para     §   \S \scriptstyle
\let\Pilcrow\P          %    \Pilcrow  ¶   \P
\mathchardef\pilcrow="227B

\mathchardef\lt="313C   %    \lt       <   <     bra
\mathchardef\gt="313E   %    \gt       >   >     ket

\let\bs\backslash       %    \bs       \
\let\us\_               %    \us       _     \_  ...

\mathchardef\lt="313C   %    \lt       <   <     bra
\mathchardef\gt="313E   %    \gt       >   >     ket

%%%% DIACRITICS %%%%%%%%%%%%%%%%%%%%%%%%%%%%%%%%%%%%%%%%%

%let\udot\d             % under-dot (text mode), frees \d
\let\odot\.             % over-dot (text mode),  frees \.
%let\hacek\v            % hacek (text mode),     frees \v
%let\makron\=           % makron (text mode),    frees \=
%let\tilda\~            % tilde (text mode),     frees \~
\let\uml\"              % umlaut (text mode),    frees \"

%def\ij/{i{\kern-.07em}j}
%def\trema#1{\discretionary{-}{#1}{\uml #1}}

%%%% amssymb %%%%%%%%%%%%%%%%%%%%%%%%%%%%%%%%%%%%%%%%%%%%

\let\le\leqslant
\let\ge\geqslant
%let\prece\preceqslant
%let\succe\succeqslant

%%%% USEFUL MISC %%%%%%%%%%%%%%%%%%%%%%%%%%%%%%%%%%%%%%%%

%def\C++{C$^{_{++}}$}

%let\writelog\wlog
%def\wl@g/{{\sc wlog}}
%def\wlog{\@ifnextchar/{\wl@g}{\writelog}}

%def\org#1{\lower1.2pt\hbox{#1}} 
% chem struct formulae: \bs, --- /  \org{C} etc. 

%%%% USEFUL INTERNAL LaTeX STUFF %%%%%%%%%%%%%%%%%%%%%%%%

%let\Ifnextchar=\@ifnextchar
%let\Ifstar=\@ifstar
%def\currsize{\@currsize}

%%%% KERNING, SPACING, BREAKING %%%%%%%%%%%%%%%%%%%%%%%%%

%def\qqquad{\hskip3em\relax}
%def\qqqquad{\hskip4em\relax}
%def\qqqqquad{\hskip5em\relax}
%def\qqqqqquad{\hskip6em\relax}
%def\qqqqqqquad{\hskip7em\relax}
%def\qqqqqqqquad{\hskip8em\relax}

%%%% LAYOUT %%%%%%%%%%%%%%%%%%%%%%%%%%%%%%%%%%%%%%%%%%%%%

%%%% COUNTERS %%%%%%%%%%%%%%%%%%%%%%%%%%%%%%%%%%%%%%%%%%%

%let\addtoreset\@addtoreset
%{A}{B} adds A to list of counters reset to 0
% when B is \refstepcounter'ed (see latex.tex)
%
%def\numbernext#1#2{\setcounter{#1}{#2}\addtocounter{#1}{\m@ne}}

%%%% EQUATIONS %%%%%%%%%%%%%%%%%%%%%%%%%%%%%%%%%%%%%%%%%%

%%%% LEMMATA %%%%%%%%%%%%%%%%%%%%%%%%%%%%%%%%%%%%%%%%%%%%

%%%% DISPLAY %%%%%%%%%%%%%%%%%%%%%%%%%%%%%%%%%%%%%%%%%%%%

%%%% MATH LAYOUT %%%%%%%%%%%%%%%%%%%%%%%%%%%%%%%%%%%%%%%%

\let\D\displaystyle
\let\T\textstyle
\let\S\scriptstyle
\let\SS\scriptscriptstyle

% array:
%def\<#1:{\begin{array}{#1}}
%def\>{\end{array}}

% array using [ ] with rounded corners:
%def\[#1:{\left\lgroup\begin{array}{#1}} 
%def\]{\end{array}\right\rgroup}

% array using ( ):
%def\(#1:{\left(\begin{array}{#1}}
%def\){\end{array}\right)}

%def\hh{\noalign{\vskip\doublerulesep}}

%%%% MATH SYMBOLS %%%%%%%%%%%%%%%%%%%%%%%%%%%%%%%%%%%%%%%

%def\d{\mathord{\rm d}}                      % d as in dx
%def\e{{\rm e}}                              % e as in e^x

%def\Ell{\hbox{\it\char`\$}}

\def\sfmath#1{{\mathchoice%
{{\sf #1}}{{\sf #1}}{{\S\sf #1}}{{\SS\sf #1}}}}
\def\Stalkset#1{\sfmath{I\kern-.12em#1}}
\def\Bset{\Stalkset B}
\def\Nset{\Stalkset N}
\def\Rset{\Stalkset R}
\def\Hset{\Stalkset H}
\def\Fset{\Stalkset F}
\def\kset{\Stalkset k}
\def\In@set{\raise.14ex\hbox{\i}\kern-.237em\raise.43ex\hbox{\i}}
\def\Roundset#1{\sfmath{\kern.14em\In@set\kern-.4em#1}}
\def\Qset{\Roundset Q}
\def\Cset{\Roundset C}
\def\Oset{\Roundset O}
\def\Zset{\sfmath{Z\kern-.44emZ}}

% \frac overwrites LaTeX's one (use TeX \over instead)
%def\fraq#1#2{{}^{#1}\!/\!{}_{\,#2}}
\def\frac#1#2{\mathord{\mathchoice%
{\T{#1\over#2}}
{\T{#1\over#2}}
{\S{#1\over#2}}
{\SS{#1\over#2}}}}
%def\half{\frac12}

\mathcode`\<="4268         % < now is \langle, \lt is <
\mathcode`\>="5269         % > now is \rangle, \gt is >

%def\biggg#1{{\hbox{$\left#1\vbox %to20.5\p@{}\right.\n@space$}}}
%def\Biggg#1{{\hbox{$\left#1\vbox %to23.5\p@{}\right.\n@space$}}}

\let\epsi=\varepsilon
\def\omikron{o}

\def\Alpha{{\rm A}}
\def\Beta{{\rm B}}
\def\Epsilon{{\rm E}}
\def\Zeta{{\rm Z}}
\def\Eta{{\rm H}}
\def\Iota{{\rm I}}
\def\Kappa{{\rm K}}
\def\Mu{{\rm M}}
\def\Nu{{\rm N}}
\def\Omikron{{\rm O}}
\def\Rho{{\rm P}}
\def\Tau{{\rm T}}
\def\Ypsilon{{\rm Y}} % differs from \Upsilon
\def\Chi{{\rm X}}

%def\dg{^{\circ}}                   % degrees

%def\1{^{-1}}                       % inverse

\def\*#1{{\bf #1}}                  % boldface e.g. vector
%def\vi{\mathord{\hbox{\bf\i}}}     % boldface vector \i
%def\vj{\mathord{\,\hbox{\bf\j}}}   % boldface vector \j

%def\union{\mathbin\cup}
%def\isect{\mathbin\cap}

%let\so\Longrightarrow
%let\oso\Longleftrightarrow
%let\os\Longleftarrow

% := and :<=>
%def\isdef{\mathrel{\smash{\stackrel{\SS\rm def}{=}}}}
%def\iffdef{\mathrel{\smash{stackrel{\SS\rm def}{\oso}}}}

\def\isdef{\mathrel{\mathop{=}\limits^{\smash{\hbox{\tiny def}}}}}
%def\iffdef{\mathrel{\mathop{\oso}\limits^{\smash{\hbox{\tiny %def}}}}}

%def\tr{\mathop{\rm tr}}            % tr[ace]
%def\ter#1{\mathop{^#1\rm ter}}     % k-ter[minant]

%let\.=\cdot
%let\x=\times                % ח (direct product)

%def\qed{ ${\S\circ}\!{}^\circ\!{\S\circ}$}
%def\qed{\vrule height 6pt width 6pt depth 0pt}

%def\edots{\mathinner{\mkern1mu
%   \raise7pt\vbox{\kern7pt\hbox{.}}\mkern1mu   %  .shorter
%   \raise4pt\hbox{.}\mkern1mu                  %     .
%   \raise1pt\hbox{.}\mkern1mu}}                %        .
%def\fdots{\mathinner{\mkern1mu
%   \raise7pt\vbox{\kern7pt\hbox{.}}            %   . ~45°
%   \raise4pt\hbox{.}                           %     .
%   \raise1pt\hbox{.}\mkern1mu}}                %       .

\def\mod#1{\allowbreak \mkern 10mu({\rm mod}\,\,#1)}
% redefines TeX's one using less space

%def\int{\intop\displaylimits}
%def\oint{\ointop\displaylimits}

%def\intoi{\int_0^1}
%def\intall{\int_{-\infty}^\infty}

%def\su#1{\mathop{\sum\raise0.7pt\hbox{$\S\!\!\!\!\!#1\,$}}}

%let\frakR\Re
%let\frakI\Im
%def\Re{\mathop{\rm Re}\nolimits}
%def\Im{\mathop{\rm Im}\nolimits}
%def\conj#1{\overline{#1\vphantom1}}
%def\cj#1{\overline{#1\vphantom+}}

%def\forAll{\mathop\forall\limits}
%def\Exists{\mathop\exists\limits}

%%%% PICTURES %%%%%%%%%%%%%%%%%%%%%%%%%%%%%%%%%%%%%%%%%%%

%def\cent{\makebox(0,0)}

%def\node{\circle*4}
%def\nOde{\circle4}

%%%% REFERENCES %%%%%%%%%%%%%%%%%%%%%%%%%%%%%%%%%%%%%%%%%

%def\opcit{[{\it op.\,cit.}]}
\def\bitem#1{\bibitem[#1]{#1}}
\def\name#1{{\sc #1}}
\def\book#1{{\sl #1\/}}
\def\paper#1{``#1''}
\def\mag#1{{\it #1\/}}
\def\vol#1{{\bf #1}}
\def\isbn#1{{\small\tt ISBN\,\,#1}}
\def\seq#1{{\small\tt #1}}
%def\url<{\verb>}
%def\@cite#1#2{[{#1\if@tempswa\ #2\fi}]}

%%%% VERBATIM CODE %%%%%%%%%%%%%%%%%%%%%%%%%%%%%%%%%%%%%%

%def\"{\verb"}

%%%% AD HOC %%%%%%%%%%%%%%%%%%%%%%%%%%%%%%%%%%%%%%%%%%%%%

%%%% WORDS %%%%%%%%%%%%%%%%%%%%%%%%%%%%%%%%%%%%%%%%%%%%%%

% \hyphenation{pre-sent pre-sents pre-sent-ed pre-sent-ing
% re-pre-sent re-pre-sents re-pre-sent-ed re-pre-sent-ing
% re-fer-ence re-fer-ences re-fer-enced re-fer-encing
% ge-o-met-ry re-la-ti-vi-ty Gauss-ian Gauss-ians
% Des-ar-gues-ian}

%def\oord/{o{\trema o}rdin\-ate}
% usage: C\oord/s, c\oord/.
% output: co\"ord... except when linebreak at co-ord...

%%%%%%%%%%%%%%%%%%%%%%%%%%%%%%%%%%%%%%%%%%%%%%%%%%%%%%%%%

                                          %
                                          %          ^ ^
\catcode`\@=12                            %          ` '
                                          %         ->*<-
                                          %           ~
\begin{document}
\PMlinkescapeword{chain}
\PMlinkescapeword{chains}
\PMlinkescapeword{embedding}
\PMlinkescapeword{embeddings}
\PMlinkescapeword{even}
\PMlinkescapeword{length}
\PMlinkescapeword{map}
\PMlinkescapeword{maps}
\PMlinkescapeword{mean}
\PMlinkescapeword{meet}
\PMlinkescapeword{odd}
\PMlinkescapeword{maximal}
\PMlinkescapeword{sound}
\PMlinkescapeword{structure}
\PMlinkescapeword{time}
\PMlinkescapeword{types}

% \subsection*{Kempe chains}

Alfred B.\ Kempe
(\PMlinkexternal{bio at St Andrews}{http://www-history.mcs.st-andrews.ac.uk/Mathematicians/Kempe.html})
first used these chains, now called after him,
in 1879 in a ``proof'' of the four-color conjecture. Although Percy Heawood
found a flaw in his proof 11 years later, the idea of Kempe chains itself is
quite sound. Heawood used it to prove 5 colors suffice for maps on the plane,
and the 1976 proof by Appel, Haken and Koch is also based on Kempe's ideas.

The original Kempe chains were used in the context of colorings of countries
on a map, or in modern terminology {\bf face colorings} of a plane graph $G$
(such that no two adjacent faces receive the same color). The idea was
extended by Heawood to embeddings of a graph in any other surface. Here
%
\begin{itemize}

\item a Kempe chain of colors $a$ and $b$ is a maximal connected set of faces
that have either of those colors. \PMlinkescapetext{Connected} as in: you can travel from any face in the set to any other, through the set. Maximal as in: there are no more faces of those colors you could enlarge the set with, i.e.\ that border the area you've already got.

\end{itemize}
%
In the modern dual formulation of the four-color theorem, faces of $G$ are
replaced by vertices of the dual graph $G^*$ (and vice versa); vertices are
adjacent (linked by an edge) in $G^*$ whenever the corresponding faces of $G$
were adjacent (sharing a border). It now becomes a {\bf vertex coloring}
problem (again, adjacent vertices must receive different colors). Here
%
\begin{itemize}

\item a Kempe chain of colors $a$ and $b$ is a maximal connected subgraph
containing only vertices of those colors. Connected as usual in graph theory
(there's a path between any two vertices) and maximal as in: there are no
more vertices of those colors you could enlarge the subgraph with, i.e.\ that
are adjacent to a vertex you've already got.

An alternative formulation: let $G(a,b)$ be the subgraph induced by all the
vertices of color $a$ or $b$ (that is, with all the edges that run between
those vertices). Now any connected component $H(a,b)$ of $G(a,b)$ is a Kempe
chain.

\end{itemize}
%
While faces imply an embedding in a surface, the vertex version of the
definition does not rely on any embedding. The chains are more useful in
the context of an embedding though.

\clearpage
\subsection*{Kempe chains of edges}

With {\bf edge coloring} of graphs (again, with different colors for adjacent
edges i.e.\ those that meet at a vertex), an analogous concept can defined.
Here
%
\begin{itemize}

\item a Kempe chain of colors $a$ and $b$ is a maximal connected subgraph
where the edges have either of those colors. Connected and maximal as before.

An alternative formulation: let $G(a,b)$ be the subgraph consisting of all the
edges of color $a$ or $b$ (with any vertices incident to them). Now any
connected component $H(a,b)$ of $G(a,b)$ is a Kempe chain.

\end{itemize}
%
While superficially similar to the previous definition, these ``Kempe chains''
are rather different animals. Their structure is far simpler. At any vertex of
$H(a,b)$, there can be at most one edge of color $a$ and one edge of color $b$
(at most two edges in all). Two things can happen:
%
\begin{itemize}

\item[$\circ$] Every vertex in the chain has two such edges. In a finite
     graph, this must mean a closed path (cycle). Note that, being colored
     alternatingly, the number of edges must now be even.

\item[$\circ$] A vertex misses out on having an edge of one of those colors,
     and the chain stops there (it can't miss out on both colors because then
     it wouldn't be part of the chain). In a finite graph, the other end of
     the chain must now also terminate somewhere (at a vertex that misses
     out either one of the colors). The chain is an open path of one or
     more edges (its length can be even or odd).

\end{itemize}

\clearpage
\subsection*{Kempe chain arguments}

One technique that can be used with all these types of chains is
%
\begin{itemize}

\item swapping the colors in one $H(a,b)$. This is always possible:
      by definition, there are no adjacent items that could
      lead to a color clash.

      Kempe slipped up when he swapped an $H(a,b)$ and an $H(a,c)$ without
      taking in account that swapping colors $a$ and $b$ in part of the
      graph could alter the shape and connectedness of any $H(a,c)$, but
      swapping colors in one Kempe chain at a time (and then taking stock of
      the lay of the land afresh) is quite safe.

      Swapping can be used to free up a color somewhere, see the
      \PMlinkid{proof of Vizing's theorem}{6932} for a repeated use
      of this ploy on edge-based Kempe chains.

\end{itemize}
%
Specific to the use in plane graphs we have that
%
\begin{itemize}

\item if a Kempe chain forms a cycle, it disconnects the sphere or plane area
      into disjoint parts (on the plane, inside and outside the cycle).

\end{itemize}
%
Specifically when four colors are used for faces:
%
\begin{itemize}

\item The whole area is divided into red/green swathes and yellow/blue ones.
      Alternatively, into red/yellow and green/blue ones. Or red/blue and
      yellow/green ones.

\end{itemize}
%
Specifically for edge-based Kempe chains in regular $\rho$-valent graphs, if
we try to edge-color the graph with only $\rho$ colors:
%
\begin{itemize}

\item No vertex can miss out on any of the $\rho$ colors, so every $H(a,b)$
      must be a cycle (of even length).

\end{itemize}
%%%%%
%%%%%
\end{document}
