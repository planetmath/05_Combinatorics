\documentclass[12pt]{article}
\usepackage{pmmeta}
\pmcanonicalname{LYMInequality}
\pmcreated{2013-03-22 14:06:01}
\pmmodified{2013-03-22 14:06:01}
\pmowner{bbukh}{348}
\pmmodifier{bbukh}{348}
\pmtitle{LYM inequality}
\pmrecord{6}{35498}
\pmprivacy{1}
\pmauthor{bbukh}{348}
\pmtype{Theorem}
\pmcomment{trigger rebuild}
\pmclassification{msc}{05D05}
\pmclassification{msc}{06A07}
\pmsynonym{LYM-inequality}{LYMInequality}
%\pmkeywords{Sperner family}
\pmrelated{SpernersTheorem}

\usepackage{amssymb}
\usepackage{amsmath}
\usepackage{amsfonts}

\newcommand*{\abs}[1]{\left\lvert #1\right\rvert}
\newcommand*{\floor}[1]{\left\lfloor #1\right\rfloor}
\makeatletter
\@ifundefined{bibname}{}{\renewcommand{\bibname}{References}}
\makeatother
\begin{document}
Let $\mathcal{F}$ be a Sperner family, that is, the collection of
subsets of $\{1,2,\dotsc,n\}$ such that no set contains any other subset.
Then
\begin{equation*}
\sum_{X\in\mathcal{F}} \frac{1}{\binom{n}{\abs{X}}}\leq 1.
\end{equation*}
This inequality is known as \emph{LYM inequality} by the names of
three people that independently discovered it:
Lubell\cite{cite:lubell_lym}, Yamamoto\cite{cite:yamamoto_lym},
Meshalkin\cite{cite:meshalkin_lym}.

Since $\binom{n}{k}\leq \binom{n}{\floor{n/2}}$ for every integer $k$, LYM inequality tells us that $\abs{F}/\binom{n}{\floor{n/2}}\leq 1$ which is \PMlinkname{Sperner's theorem}{SpernersTheorem}.

\begin{thebibliography}{1}

\bibitem{cite:engel_sperner}
Konrad Engel.
\newblock {\em Sperner theory}, volume~65 of {\em Encyclopedia of Mathematics
  and Its Applications}.
\newblock Cambridge University Press.
\newblock \PMlinkexternal{Zbl 0868.05001}{http://www.emis.de/cgi-bin/zmen/ZMATH/en/quick.html?type=html&an=0868.05001}.

\bibitem{cite:lubell_lym}
David Lubell.
\newblock A short proof of Sperner's lemma.
\newblock {\em J. Comb. Theory}, 1:299, 1966.
\newblock \PMlinkexternal{Zbl 0151.01503}{http://www.emis.de/cgi-bin/zmen/ZMATH/en/quick.html?type=html&an=0151.01503}.

\bibitem{cite:meshalkin_lym}
Lev~D. Meshalkin.
\newblock Generalization of Sperner's theorem on the number of subsets of a
  finite set.
\newblock {\em Teor. Veroyatn. Primen.}, 8:219--220, 1963.
\newblock \PMlinkexternal{Zbl 0123.36303}{http://www.emis.de/cgi-bin/zmen/ZMATH/en/quick.html?type=html&an=0123.36303}.

\bibitem{cite:yamamoto_lym}
Koichi Yamamoto.
\newblock Logarithmic order of free distributive lattice.
\newblock {\em J. Math. Soc. Japan}, 6:343--353, 1954.
\newblock \PMlinkexternal{Zbl 0056.26301}{http://www.emis.de/cgi-bin/zmen/ZMATH/en/quick.html?type=html&an=0056.26301}.

\end{thebibliography}

%@ARTICLE{cite:lubell_lym,
% author    = {David Lubell},
% title     = {A short proof of {S}perner's lemma},
% journal   = {J. Comb. Theory},
% volume    = {1},
% year      = {1966},
% pages     = {299},
% note      = {\PMlinkexternal{Zbl %0151.01503}{http://www.emis.de/cgi-bin/zmen/ZMATH/en/quick.html?type=html&an=0151.01503}}
%}
%
%@ARTICLE{cite:meshalkin_lym,
% author    = {Lev D. Meshalkin},
% title     = {Generalization of {S}perner's theorem on the number of subsets of a %finite set},
% journal   = {Teor. Veroyatn. Primen.},
% volume    = {8},
% year      = {1963},
% pages     = {219--220},
% note      = {\PMlinkexternal{Zbl %0123.36303}{http://www.emis.de/cgi-bin/zmen/ZMATH/en/quick.html?type=html&an=0123.36303}}
%}
%
%@ARTICLE{cite:yamamoto_lym,
% author    = {Koichi Yamamoto},
% title     = {Logarithmic order of free distributive lattice},
% journal   = {J. Math. Soc. Japan},
% volume    = {6},
% year      = {1954},
% pages     = {343--353},
% note      = {\PMlinkexternal{Zbl %0056.26301}{http://www.emis.de/cgi-bin/zmen/ZMATH/en/quick.html?type=html&an=0056.26301}}
%}
%%%%%
%%%%%
\end{document}
