\documentclass[12pt]{article}
\usepackage{pmmeta}
\pmcanonicalname{ExpectationExample}
\pmcreated{2013-11-05 17:59:13}
\pmmodified{2013-11-05 17:59:13}
\pmowner{pahio}{2872}
\pmmodifier{pahio}{2872}
\pmtitle{expectation example}
\pmrecord{2}{87656}
\pmprivacy{1}
\pmauthor{pahio}{2872}
\pmtype{Example}
\pmclassification{msc}{05A18}
\pmclassification{msc}{26E60}
\pmclassification{msc}{60A05}

% this is the default PlanetMath preamble.  as your knowledge
% of TeX increases, you will probably want to edit this, but
% it should be fine as is for beginners.

% almost certainly you want these
\usepackage{amssymb}
\usepackage{amsmath}
\usepackage{amsfonts}

% need this for including graphics (\includegraphics)
\usepackage{graphicx}
% for neatly defining theorems and propositions
\usepackage{amsthm}

% making logically defined graphics
%\usepackage{xypic}
% used for TeXing text within eps files
%\usepackage{psfrag}

% there are many more packages, add them here as you need them

% define commands here

\begin{document}
The contraharmonic mean of several positive numbers 
$u_1$, $u_2$, $\ldots$, $u_n$  is defined as
$$c \;:=\; \frac{u_1^2\!+\!u_2^2\!+\ldots+\!u_n^2}{u_1\!+\!u_2\!+\ldots+\!u_n}.$$                                        
This \PMlinkescapetext{concept} has certain applications; one 
of them is by [1] the following.

If $\langle u_1$, $u_2$, $\ldots$, $u_n\rangle$ is the distribution of the seats of $n$ parties, 
$s$ the total number of seats in the body of delegates 
($u_1\!+\!u_2\!+\ldots+\!u_n\,=\,s$), 
and one draws a random seat (with probability $1/s$), then the size of the drawn 
delegate{'}s party has the {\it expected value}
$$\frac{u_1}{s}\!\cdot\!u_1+\frac{u_2}{s}\!\cdot\!u_2+\ldots+\frac{u_n}{s}\!\cdot\!u_n 
\;=\; c.$$

\begin{thebibliography}{8}
\bibitem{C}{\sc Caulier, Jean-Fran\c cois}:  The interpretation of the Laakso{--}Taagepera effective number of parties.\;   {--} {\it Documents de travail du Centre d{'}Economie de la Sorbonne} (2011.06).
\end{thebibliography}


\end{document}
