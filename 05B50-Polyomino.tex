\documentclass[12pt]{article}
\usepackage{pmmeta}
\pmcanonicalname{Polyomino}
\pmcreated{2013-03-22 15:20:18}
\pmmodified{2013-03-22 15:20:18}
\pmowner{s0}{9826}
\pmmodifier{s0}{9826}
\pmtitle{polyomino}
\pmrecord{10}{37156}
\pmprivacy{1}
\pmauthor{s0}{9826}
\pmtype{Definition}
\pmcomment{trigger rebuild}
\pmclassification{msc}{05B50}
\pmdefines{n-omino}
\pmdefines{domino}
\pmdefines{tromino}
\pmdefines{tetromino}
\pmdefines{fixed polyomino}
\pmdefines{lattice animal}

\endmetadata

% this is the default PlanetMath preamble.  as your knowledge
% of TeX increases, you will probably want to edit this, but
% it should be fine as is for beginners.

% almost certainly you want these
\usepackage{amssymb}
\usepackage{amsmath}
\usepackage{amsfonts}

% used for TeXing text within eps files
%\usepackage{psfrag}
% need this for including graphics (\includegraphics)
\usepackage[dvips]{graphics}
% for neatly defining theorems and propositions
%\usepackage{amsthm}
% making logically defined graphics
%%%\usepackage{xypic}

% there are many more packages, add them here as you need them

% define commands here
\begin{document}
A polyomino consists of a number of identical connected squares placed
in distinct locations in the plane so that at least one side of each
square is adjacent to (i.e. completely coincides with the side of)
another square (if the polyomino consists of at least two squares).

A polyomino with $n$ squares is called an {\em n-omino}. For small $n$,
polyominoes have special names. A 1-omino is called a {\em monomino},
a 2-omino a {\em domino}, a 3-omino a {\em tromino} or {\em triomino},
etc. The famous Tetris video game derives its name from the fact that
the bricks are {\em tetrominoes} or 4-ominoes.

\begin{figure}[h]
\includegraphics{Polyomino.1.eps}

\includegraphics{Polyomino.2.eps}

\includegraphics{Polyomino.3.eps}

\includegraphics{Polyomino.4.eps}

\includegraphics{Polyomino.5.eps}
\sf\caption{All distinct 1-, 2-, 3-, 4-, and 5-ominoes. Pentominoes
have been scaled in the figure to fit on the page.}
\end{figure}

{\em Fixed polyominoes} (which are also called {\em lattice animals})
are considered distinct if they cannot be translated into each other,
while {\em free polyominoes} must also be distinct under rotation and
reflection.

\begin{figure}[h]
\includegraphics{Polyomino.6.eps}

\includegraphics{Polyomino.7.eps}
\sf\caption{All distinct, fixed dominoes and trominoes.}
\end{figure}

The topic of how many distinct (free or fixed) n-ominoes exist for a given
$n$ has been the subject of much research. It is known that the
number of free n-ominoes $A_n$ grows exponentially. More precisely, it
can be proven that $3.72^n < A_n < 4.65^n$.

Polyominoes are special instances of {\em polyforms}.
%%%%%
%%%%%
\end{document}
