\documentclass[12pt]{article}
\usepackage{pmmeta}
\pmcanonicalname{MultiindexNotation}
\pmcreated{2013-03-22 13:41:32}
\pmmodified{2013-03-22 13:41:32}
\pmowner{matte}{1858}
\pmmodifier{matte}{1858}
\pmtitle{multi-index notation}
\pmrecord{15}{34366}
\pmprivacy{1}
\pmauthor{matte}{1858}
\pmtype{Definition}
\pmcomment{trigger rebuild}
\pmclassification{msc}{05-00}
\pmdefines{multi-index}
\pmdefines{multi-indices}

% this is the default PlanetMath preamble.  as your knowledge
% of TeX increases, you will probably want to edit this, but
% it should be fine as is for beginners.

% almost certainly you want these
\usepackage{amssymb}
\usepackage{amsmath}
\usepackage{amsfonts}

% used for TeXing text within eps files
%\usepackage{psfrag}
% need this for including graphics (\includegraphics)
%\usepackage{graphicx}
% for neatly defining theorems and propositions
%\usepackage{amsthm}
% making logically defined graphics
%%%\usepackage{xypic}

% there are many more packages, add them here as you need them

% define commands here

\newcommand{\sR}[0]{\mathbb{R}}
\newcommand{\sC}[0]{\mathbb{C}}
\newcommand{\sN}[0]{\mathbb{N}}
\newcommand{\sZ}[0]{\mathbb{Z}}
\begin{document}
Multi-indices form a powerful notational device for keeping track
of multiple derivatives or multiple powers. In many respects 
these resemble natural numbers. 
For example, one can define the factorial, binomial coefficients,
and derivatives for multi-indices. 
Using these one can state traditional results such as the
multinomial theorem, 
Leibniz' rule, Taylor's formula, etc. 
very concisely. In fact, the multi-dimensional results are more or
 less obtained simply by  replacing usual indices in $\sN$ with multi-indices. 
See below for examples.

{\bf Definition}
A \emph{multi-index} is an $n$-tuple 
$\alpha=(\alpha_1,\ldots, \alpha_n)$ of non-negative integers $\alpha_1,\ldots, \alpha_n$. In other words,
$\alpha \in \sN^n$. Usually, $n$ is the dimension of the underlying space.
Therefore, when dealing with multi-indices, $n$ is usually 
assumed clear from the context.

\subsubsection*{Operations on multi-indices}
For a multi-index $\alpha$, we define the \emph{length} (or \emph{order}) as
$$ 
  |\alpha| = \alpha_1+\cdots + \alpha_n,
$$
and the \emph{factorial}  as
$$ 
  \alpha! = \prod_{k=1}^n \alpha_k!.
$$
If $\alpha=(\alpha_1,\ldots, \alpha_n)$ and 
$\beta=(\beta_1,\ldots, \beta_n)$ are two multi-indices, 
their sum and difference is defined  component-wise as
\begin{eqnarray*}
 \alpha+\beta &=&(\alpha_1+\beta_1, \ldots, \alpha_n+\beta_n),\\
 \alpha-\beta &=&(\alpha_1-\beta_1, \ldots, \alpha_n-\beta_n).
\end{eqnarray*}
Thus $|\alpha\pm \beta| = |\alpha|\pm |\beta|$.
Also, if $\beta_k\le \alpha_k$ for all $k=1,\ldots, n$, then we write 
$\beta\le \alpha$. For multi-indices $\alpha,\beta$, with  
$\beta\le \alpha$, we define
$$ 
  {\alpha\choose \beta} = \frac{\alpha!}{(\alpha-\beta)! \beta!}.
$$
For a point $x=(x_1,\ldots, x_n)$ in $\mathbb{R}^n$ (with 
standard coordinates) we define 
$$ 
  x^\alpha= \prod_{k=1}^n x_k^{\alpha_k}.
$$
Also, if $f\colon \mathbb{R}^n \to \mathbb{R}$ is a smooth function, and
$\alpha=(\alpha_1,\ldots, \alpha_n)$ is a multi-index, we define 
$$   
  \partial^\alpha f = \frac{\partial^{|\alpha|}}{\partial^{\alpha_1}e_1 \cdots \partial^{\alpha_n}e_n} f,
$$
where $e_1,\ldots, e_n$ are the standard unit vectors of $\mathbb{R}^n$. 
Since $f$ is sufficiently smooth, the order in which the derivations are 
performed is irrelevant. For multi-indices $\alpha$ and $\beta$, we thus
have 
$$ 
  \partial^\alpha \partial^\beta = \partial^{\alpha+\beta} = \partial^{\beta+\alpha} = \partial^\beta \partial^\alpha.
$$

\subsubsection*{Examples}
\begin{enumerate}
\item If $n$ is a positive integer, and $x_1,\ldots, x_k$ are
complex numbers, the multinomial expansion  states that
$$ 
  (x_1+\cdots + x_k)^n = n! \sum_{|\alpha|=n} \frac{x^\alpha}{\alpha!}, 
$$
where $x=(x_1,\ldots, x_k)$ and $\alpha$ is a multi-index.
(\PMlinkname{proof}{MultinomialTheoremProof})
\item Leibniz' rule: If $f,g\colon\sR^n \to \sR$ are smooth functions, and $\beta$ is
a multi-index, then 
$$ \partial^\beta(fg) = \sum_{\alpha\le \beta} {\beta \choose \alpha} \partial^\alpha(f)\, \partial^{\beta-\alpha}(g),$$
where $\alpha$ is a multi-index.
\end{enumerate}

\begin{thebibliography}{9}
\bibitem{reedsimon} M. Reed, B. Simon, \emph{Methods of Mathematical Physics,
I - Functional Analysis}, Academic Press, 1980.
\end{thebibliography}
%%%%%
%%%%%
\end{document}
