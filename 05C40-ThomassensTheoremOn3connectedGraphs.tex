\documentclass[12pt]{article}
\usepackage{pmmeta}
\pmcanonicalname{ThomassensTheoremOn3connectedGraphs}
\pmcreated{2013-03-22 13:11:09}
\pmmodified{2013-03-22 13:11:09}
\pmowner{Mathprof}{13753}
\pmmodifier{Mathprof}{13753}
\pmtitle{Thomassen's theorem on $3$-connected graphs}
\pmrecord{6}{33634}
\pmprivacy{1}
\pmauthor{Mathprof}{13753}
\pmtype{Theorem}
\pmcomment{trigger rebuild}
\pmclassification{msc}{05C40}
\pmrelated{EdgeContraction}

% this is the default PlanetMath preamble.  as your knowledge
% of TeX increases, you will probably want to edit this, but
% it should be fine as is for beginners.

% almost certainly you want these
\usepackage{amssymb}
\usepackage{amsmath}
\usepackage{amsfonts}

% used for TeXing text within eps files
%\usepackage{psfrag}
% need this for including graphics (\includegraphics)
%\usepackage{graphicx}
% for neatly defining theorems and propositions
%\usepackage{amsthm}
% making logically defined graphics
%%%\usepackage{xypic}

% there are many more packages, add them here as you need them

% define commands here
\begin{document}
Every \PMlinkname{$3$-connected}{KConnectedGraph} graph $G$ with more
than 4 vertices has an edge $e$ such that \PMlinkname{$G/e$}{EdgeContraction} is also $3$-connected.

Note: $G/e$ denotes the graph obtained from $G$ by contracting the edge $e$.
If $e=xy$ we also use the notation $G/xy$.

Suppose such an edge doesn't exist. Then, for every edge $e=xy$, the
graph $G/e$ isn't $3$-connected and can be made disconnected by
removing 2 vertices. Since $\kappa(G)\geq 3$, our contracted vertex
$v_{xy}$ has to be one of these two. So for every edge $e$, $G$ has a
vertex $z\neq x,y$ such that $\{v_{xy},z\}$ separates $G/e$.  Any 2
vertices separated by $\{v_{xy},z\}$ in $G/e$ are separated in $G$ by
$S:=\{x,y,z\}$. Since the minimal size of a separating set is 3, every
vertex in $S$ has an adjacent vertex in every component of $G-S$.

Now we choose the edge $e$, the vertex $z$ and the component $C$ such
that $|C|$ is minimal. We also choose a vertex $v$ adjacent to $z$ in
$C$.

By construction $G/zv$ is not $3$-connected since removing $xy$
disconnects $C-v$ from $G/zv$. So there is a vertex $w$ such that
$\{z,v,w\}$ separates $G$ and as above every vertex in $\{z,v,w\}$ has
an adjacent vertex in every component of $G-\{z,v,w\}$. We now
consider a component $D$ of $G-\{z,v,w\}$ that doesn't contain $x$ or
$y$. Such a component exists since $x$ and $y$ belong to the same
component and $G-\{z,v,w\}$ isn't connected. Any vertex adjacent to
$v$ in $D$ is also an element of $C$ since $v$ is an element of
$C$. This means $D$ is a proper subset of $C$ which contradicts our
assumption that $|C|$ was minimal.
%%%%%
%%%%%
\end{document}
