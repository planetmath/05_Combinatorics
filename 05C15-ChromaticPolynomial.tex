\documentclass[12pt]{article}
\usepackage{pmmeta}
\pmcanonicalname{ChromaticPolynomial}
\pmcreated{2013-03-22 13:58:25}
\pmmodified{2013-03-22 13:58:25}
\pmowner{bbukh}{348}
\pmmodifier{bbukh}{348}
\pmtitle{chromatic polynomial}
\pmrecord{8}{34745}
\pmprivacy{1}
\pmauthor{bbukh}{348}
\pmtype{Definition}
\pmcomment{trigger rebuild}
\pmclassification{msc}{05C15}
\pmclassification{msc}{05B35}
%\pmkeywords{matroid}
%\pmkeywords{coloring}
\pmrelated{Matroid}
\pmrelated{ChromaticNumber}
\pmdefines{Tutte polynomial}

\usepackage{amssymb}
\usepackage{amsmath}
\usepackage{amsfonts}
\newcommand{\N}{\mathbb{N}}
\newcommand{\Z}{\mathbb{Z}}
\begin{document}
\PMlinkescapeword{rank}
\PMlinkescapeword{inversion}
\PMlinkescapeword{contains}
\PMlinkescapeword{information}
Let $G$ be a graph (in the sense of graph theory) whose set $V$
of vertices is finite and nonempty, and which has no loops or
multiple edges.
For any natural number $x$, let $\chi(G,x)$, or just $\chi(x)$,
denote the number of $x$-colorations of $G$, i.e. the number of
mappings $f\colon V\to\{1,2,\ldots,x\}$ such that $f(a)\ne f(b)$ for any
pair $(a,b)$ of adjacent vertices.
Let us prove that $\chi$ (which is called the
\emph{chromatic polynomial} of the graph $G$)
is a polynomial function in $x$ with coefficients in $\Z$.
Write $E$ for the set of edges in $G$. If $|E|$=0, then trivially
$\chi(x)=x^{|V|}$ (where $|\cdot|$ denotes the number of elements
of a finite set).
If not, then we choose an edge $e$ and construct two graphs
having fewer edges than $G$: $H$ is obtained from $G$ by contracting
the edge $e$, and $K$ is obtained from $G$ by omitting the edge $e$.
We have
\begin{equation} \label{eq:ind}
\chi(G,x)=\chi(K,x)-\chi(H,x)
\end{equation}
for all $x\in\N$, because the polynomial $\chi(K,x)$ is the number
of colorations of the vertices of $G$ which might or might not be
valid for the edge $e$, while $\chi(H,x)$ is the number which
are not valid.
By induction on $|E|$, \eqref{eq:ind} shows that $\chi(G,x)$ is
a polynomial over $\Z$.

By refining the argument a little, one can show
$$\chi(x)=x^{|V|}-|E|x^{|V|-1}+\ldots\pm sx^k\;,$$
for some nonzero integer $s$, where $k$ is the number
of connected components of $G$, and the coefficients alternate in sign.

With the help of the M\"obius-Rota
inversion formula (see Moebius Inversion),
or directly by induction, one can prove
$$\chi(x)=\sum_{F\subset E}(-1)^{|F|}x^{|V|-r(F)}$$
where the sum is over all subsets $F$ of $E$, and $r(F)$ denotes the
rank of $F$ in $G$, i.e. the number of elements of
any maximal cycle-free subset of $F$.
(Alternatively, the sum may be taken only over subsets $F$ such that
$F$ is equal to the span of $F$; all other summands cancel
out in pairs.)

The \emph{chromatic number} of $G$ is the smallest $x>0$ such that
$\chi(G,x)>0$ or, equivalently, such that $\chi(G,x)\ne0$.

The \emph{Tutte polynomial} of a graph, or more generally of a matroid
$(E,r)$, is this function of two variables:
$$t(x,y)=\sum_{F\subset E}(x-1)^{r(E)-r(F)}(y-1)^{|F|-r(F)}\;.$$
Compared to the chromatic polynomial, the Tutte contains more information
about the matroid.
Still, two or more nonisomorphic matroids may have the same Tutte polynomial.
%%%%%
%%%%%
\end{document}
