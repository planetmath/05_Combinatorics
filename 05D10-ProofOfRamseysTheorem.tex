\documentclass[12pt]{article}
\usepackage{pmmeta}
\pmcanonicalname{ProofOfRamseysTheorem}
\pmcreated{2013-03-22 12:55:52}
\pmmodified{2013-03-22 12:55:52}
\pmowner{mathcam}{2727}
\pmmodifier{mathcam}{2727}
\pmtitle{proof of Ramsey's theorem}
\pmrecord{8}{33286}
\pmprivacy{1}
\pmauthor{mathcam}{2727}
\pmtype{Proof}
\pmcomment{trigger rebuild}
\pmclassification{msc}{05D10}

\endmetadata

% this is the default PlanetMath preamble.  as your knowledge
% of TeX increases, you will probably want to edit this, but
% it should be fine as is for beginners.

% almost certainly you want these
\usepackage{amssymb}
\usepackage{amsmath}
\usepackage{amsfonts}

% used for TeXing text within eps files
%\usepackage{psfrag}
% need this for including graphics (\includegraphics)
%\usepackage{graphicx}
% for neatly defining theorems and propositions
%\usepackage{amsthm}
% making logically defined graphics
%%%\usepackage{xypic}

% there are many more packages, add them here as you need them

% define commands here
%\PMlinkescapeword{theory}
\begin{document}
\PMlinkescapeword{simple}
\PMlinkescapeword{state}
\PMlinkescapeword{states}

$$\omega\rightarrow(\omega)^n_k$$
is proven by induction on $n$.

If $n=1$ then this just states that any partition of an infinite set into a finite number of subsets must include an infinite set; that is, the union of a finite number of finite sets is finite.  This is simple enough to prove: since there are a finite number of sets, there is a largest set of size $x$.  Let the number of sets be $y$.  Then the size of the union is no more than $xy$.

If
$$\omega\rightarrow(\omega)^n_k$$

then we can show that 

$$\omega\rightarrow(\omega)^{n+1}_k$$

Let $f$ be some coloring of $[S]^{n+1}$ by $k$ where $S$ is an infinite subset of $\omega$.  Observe that, given an $x<\omega$, we can define $f^x\colon[S\setminus\{x\}]^{n}\rightarrow k$ by $f^x(X)=f(\{x\}\cup X)$.  Since $S$ is infinite, by the induction hypothesis this will have an infinite homogeneous set.

Then we define a sequence of integers $\langle n_i\rangle_{i\in\omega}$ and a sequence of infinite subsets of $\omega$, $\langle S_i\rangle_{i\in\omega}$ by induction.  Let $n_0=0$ and let $S_0=\omega$.  Given $n_i$ and $S_i$ for $i\leq j$ we can define $S_{j}$ as an infinite homogeneous set for $f^{n_i}\colon[S_{j-1}]^n\rightarrow k$ and $n_j$ as the least element of $S_j$.

Obviously $N=\bigcup \{n_i\}$ is infinite, and it is also homogeneous, since each $n_i$ is contained in $S_j$ for each $j\leq i$.
%%%%%
%%%%%
\end{document}
