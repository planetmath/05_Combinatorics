\documentclass[12pt]{article}
\usepackage{pmmeta}
\pmcanonicalname{CountingCompositionsOfAnInteger}
\pmcreated{2013-03-22 17:37:55}
\pmmodified{2013-03-22 17:37:55}
\pmowner{rm50}{10146}
\pmmodifier{rm50}{10146}
\pmtitle{counting compositions of an integer}
\pmrecord{5}{40053}
\pmprivacy{1}
\pmauthor{rm50}{10146}
\pmtype{Result}
\pmcomment{trigger rebuild}
\pmclassification{msc}{05-00}
\pmdefines{composition}

% this is the default PlanetMath preamble.  as your knowledge
% of TeX increases, you will probably want to edit this, but
% it should be fine as is for beginners.

% almost certainly you want these
\usepackage{amssymb}
\usepackage{amsmath}
\usepackage{amsfonts}

% used for TeXing text within eps files
%\usepackage{psfrag}
% need this for including graphics (\includegraphics)
%\usepackage{graphicx}
% for neatly defining theorems and propositions
%\usepackage{amsthm}
% making logically defined graphics
%%%\usepackage{xypic}

% there are many more packages, add them here as you need them

% define commands here

\begin{document}
\PMlinkescapeword{place}
\PMlinkescapeword{even}
\PMlinkescapeword{section}
\PMlinkescapeword{sections}
\PMlinkescapeword{composition}
A \emph{composition} of a nonnegative integer $n$ is a sequence $(a_1,\ldots,a_k)$ of positive integers with $\sum a_i=n$. Denote by $C_n$ the number of compositions of $n$, and denote by $S_n$ the set of those compositions. (Note that this is a very different - and simpler - concept than the number of partitions of an integer; here the \PMlinkname{order}{PartialOrder} matters).

For some small values of $n$, we have
\begin{align*}
C_0&=1\\
C_1 &= 1\\
C_2 &= 2 \qquad (2), (1,1)\\
C_3 &= 4 \qquad (3), (1,2), (2,1), (1,1,1)
\end{align*}
In fact, it is easy to see that $C_n=2C_{n-1}$ for $n>1$: each composition $(a_1,\ldots,a_k)$ of $n-1$ can be associated with two different compositions of $n$
\begin{gather*}
(a_1,a_2,\ldots,a_k,1)\\
(a_1,a_2,\ldots,a_k+1)
\end{gather*}
We thus get a map $\varphi:S_{n-1}\times\{0,1\}\to S_n$ given by
\begin{gather*}
\varphi((a_1,\ldots,a_k),0)=(a_1,\ldots,a_k,1)\\
\varphi((a_1,\ldots,a_k),1)=(a_1,\ldots,a_k+1)
\end{gather*}
and this map is clearly injective. But it is also clearly surjective, for given $(a_1,\ldots,a_k)\in S_n$, if $a_k=1$ then the composition is the image of $((a_1,\ldots,a_{k-1}),0)$ while if $a_k>1$, then it is the image of $((a_1,\ldots,a_{k-1}),1)$. This proves that (for $n>1$) $C_n = 2C_{n-1}$.

We can also figure out how many compositions there are of $n$ with $k$ parts. Think of a box with $n$ sections in it, with dividers between each pair of sections and a chip in each section; there are thus $n$ chips and $n-1$ dividers. If we leave $k-1$ of the dividers in place, the result is a composition of $n$ with $k$ parts; there are obviously $\dbinom{n-1}{k-1}$ ways to do this, so the number of compositions of $n$ into $k$ parts is simply $\dbinom{n-1}{k-1}$. Note that this gives even a simpler proof of the first result, since
\[\sum_{k=1}^n\dbinom{n-1}{k-1} = \sum_{k=0}^{n-1}\dbinom{n-1}{k}=2^{n-1}\]
%%%%%
%%%%%
\end{document}
