\documentclass[12pt]{article}
\usepackage{pmmeta}
\pmcanonicalname{closedWalkTrekTrailPath}
\pmcreated{2013-03-22 15:09:50}
\pmmodified{2013-03-22 15:09:50}
\pmowner{marijke}{8873}
\pmmodifier{marijke}{8873}
\pmtitle{(closed) walk / trek / trail / path}
\pmrecord{9}{36913}
\pmprivacy{1}
\pmauthor{marijke}{8873}
\pmtype{Definition}
\pmcomment{trigger rebuild}
\pmclassification{msc}{05C38}
\pmrelated{Path2}
\pmrelated{ConnectedGraph}
\pmrelated{KConnectedGraph}
\pmrelated{Diameter3}
\pmdefines{walk}
\pmdefines{trek}
\pmdefines{trail}
\pmdefines{path}
\pmdefines{chain}
\pmdefines{circuit}
\pmdefines{cycle}
\pmdefines{closed walk}
\pmdefines{closed trek}
\pmdefines{closed trail}
\pmdefines{closed path}
\pmdefines{closed chain}
\pmdefines{open walk}
\pmdefines{open trek}
\pmdefines{open trail}
\pmdefines{open path}
\pmdefines{open chain}
\pmdefines{${s}$-arc}
\pmdefines{${s}$-cycle}
\pmdefines{elementary cycle}
\pmdefines{${s}$-circuit}

\usepackage{amssymb}

% makra.sty 1989-2005 by Marijke van Gans %
                                          %          ^ ^
\catcode`\@=11                            %          o o
                                          %         ->*<-
                                          %           ~
%%%% CHARS %%%%%%%%%%%%%%%%%%%%%%%%%%%%%%%%%%%%%%%%%%%%%%

                        %    code    char  frees  for

%let\Para\S             %    \Para     §   \S \scriptstyle
%let\Pilcrow\P          %    \Pilcrow  ¶   \P

\mathchardef\lt="313C   %    \lt       <   <     bra
\mathchardef\gt="313E   %    \gt       >   >     ket

%let\bs\backslash       %    \bs       \
%let\us\_               %    \us       _     \_    display

%%%% DIACRITICS %%%%%%%%%%%%%%%%%%%%%%%%%%%%%%%%%%%%%%%%%

%let\udot\d             % under-dot (text mode), frees \d
%let\odot\.             % over-dot (text mode),  frees \.
%let\hacek\v            % hacek (text mode),     frees \v
%let\makron\=           % makron (text mode),    frees \=
%let\tilda\~            % tilde (text mode),     frees \~
%let\uml\"              % umlaut (text mode),    frees \"

%def\ij/{i{\kern-.07em}j}
%def\trema#1{\discretionary{-}{#1}{\uml #1}}

%%%% amssymb %%%%%%%%%%%%%%%%%%%%%%%%%%%%%%%%%%%%%%%%%%%%

\let\le\leqslant
\let\ge\geqslant
%let\prece\preceqslant
%let\succe\succeqslant

%%%% USEFUL MISC %%%%%%%%%%%%%%%%%%%%%%%%%%%%%%%%%%%%%%%%

%def\C++{C$^{_{++}}$}
 
%let\writelog\wlog
%def\wl@g/{{\sc wlog}}
%def\wlog{\@ifnextchar/{\wl@g}{\writelog}}

%def\org#1{\lower1.2pt\hbox{#1}} 
% chem struct formulae: \bs, --- /  \org{C} etc. 

%%%% USEFUL INTERNAL LaTeX STUFF %%%%%%%%%%%%%%%%%%%%%%%%

%let\Ifnextchar=\@ifnextchar
%let\Ifstar=\@ifstar
%def\currsize{\@currsize}

%%%% KERNING, SPACING, BREAKING %%%%%%%%%%%%%%%%%%%%%%%%%

%def\qqquad{\hskip3em\relax}
%def\qqqquad{\hskip4em\relax}
%def\qqqqquad{\hskip5em\relax}
%def\qqqqqquad{\hskip6em\relax}
%def\qqqqqqquad{\hskip7em\relax}
%def\qqqqqqqquad{\hskip8em\relax}

%%%% LAYOUT %%%%%%%%%%%%%%%%%%%%%%%%%%%%%%%%%%%%%%%%%%%%%

%%%% COUNTERS %%%%%%%%%%%%%%%%%%%%%%%%%%%%%%%%%%%%%%%%%%%

%%%% EQUATIONS %%%%%%%%%%%%%%%%%%%%%%%%%%%%%%%%%%%%%%%%%%

%%%% LEMMATA %%%%%%%%%%%%%%%%%%%%%%%%%%%%%%%%%%%%%%%%%%%%

%%%% DISPLAY %%%%%%%%%%%%%%%%%%%%%%%%%%%%%%%%%%%%%%%%%%%%

%%%% MATH LAYOUT %%%%%%%%%%%%%%%%%%%%%%%%%%%%%%%%%%%%%%%%

\let\D\displaystyle
\let\T\textstyle
\let\S\scriptstyle
\let\SS\scriptscriptstyle

% array:
%def\<#1:{\begin{array}{#1}}
%def\>{\end{array}}

% array using [ ] with rounded corners:
%def\[#1:{\left\lgroup\begin{array}{#1}} 
%def\]{\end{array}\right\rgroup}

% array using ( ):
%def\(#1:{\left(\begin{array}{#1}}
%def\){\end{array}\right)}

% \def\hh{\noalign{\vskip\doublerulesep}}

%%%% MATH SYMBOLS %%%%%%%%%%%%%%%%%%%%%%%%%%%%%%%%%%%%%%%

%def\d{\mathord{\rm d}}                      % d as in dx
%def\e{{\rm e}}                              % e as in e^x

%def\Ell{\hbox{\it\char`\$}}

\def\sfmath#1{{\mathchoice%
{{\sf #1}}{{\sf #1}}{{\S\sf #1}}{{\SS\sf #1}}}}
\def\Stalkset#1{\sfmath{I\kern-.12em#1}}
%def\Bset{\Stalkset B}
\def\Nset{\Stalkset N}
%def\Pset{\Stalkset P}
\def\Rset{\Stalkset R}
%def\Hset{\Stalkset H}
%def\Fset{\Stalkset F}
%def\Kset{\Stalkset K}
%def\kset{\Stalkset k}
\def\In@set{\raise.14ex\hbox{\i}\kern-.237em\raise.43ex\hbox{\i}}
\def\Roundset#1{\sfmath{\kern.14em\In@set\kern-.4em#1}}
\def\Qset{\Roundset Q}
\def\Cset{\Roundset C}
%def\Oset{\Roundset O}
%def\Gset{\Roundset G}
\def\Zset{\sfmath{Z\kern-.44emZ}}

%frac overwrites LaTeX's one (use TeX \over instead)
%def\fraq#1#2{{}^{#1}\!/\!{}_{\,#2}}
\def\frac#1#2{\mathord{\mathchoice%
{\T{#1\over#2}}
{\T{#1\over#2}}
{\S{#1\over#2}}
{\SS{#1\over#2}}}}

%def\half{\frac12}

\mathcode`\<="4268         % < now is \langle, \lt is <
\mathcode`\>="5269         % > now is \rangle, \gt is >

%def\biggg#1{{\hbox{$\left#1\vbox %to20.5\p@{}\right.\n@space$}}}
%def\Biggg#1{{\hbox{$\left#1\vbox %to23.5\p@{}\right.\n@space$}}}

\let\epsi=\varepsilon
\def\omikron{o}

\def\Alpha{{\rm A}}
\def\Beta{{\rm B}}
\def\Epsilon{{\rm E}}
\def\Zeta{{\rm Z}}
\def\Eta{{\rm H}}
\def\Iota{{\rm I}}
\def\Kappa{{\rm K}}
\def\Mu{{\rm M}}
\def\Nu{{\rm N}}
\def\Omikron{{\rm O}}
\def\Rho{{\rm P}}
\def\Tau{{\rm T}}
\def\Ypsilon{{\rm Y}} % differs from \Upsilon
\def\Chi{{\rm X}}

%def\dg{^{\circ}}                   % degrees

%def\1{^{-1}}                       % inverse

%def\*#1{{\bf #1}}                  % boldface e.g. vector
%def\vi{\mathord{\hbox{\bf\i}}}     % boldface vector \i
%def\vj{\mathord{\,\hbox{\bf\j}}}   % boldface vector \j

%def\union{\mathbin\cup}
%def\isect{\mathbin\cap}

%let\so\Longrightarrow
%let\oso\Longleftrightarrow
%let\os\Longleftarrow

% := and :<=>
%def\isdef{\mathrel{\smash{\stackrel{\SS\rm def}{=}}}}
%def\iffdef{\mathrel{\smash{stackrel{\SS\rm def}{\oso}}}}

%def\isdef{\mathrel{\mathop{=}\limits^{\smash{\hbox{\tiny def}}}}}
%def\iffdef{\mathrel{\mathop{\oso}\limits^{\smash{\hbox{\tiny def}}}}}

%def\tr{\mathop{\rm tr}}            % tr[ace]
%def\ter#1{\mathop{^#1\rm ter}}     % k-ter[minant]

%let\.=\cdot
%let\x=\times                % ח (direct product)

%def\qed{ ${\S\circ}\!{}^\circ\!{\S\circ}$}
%def\qed{\vrule height 6pt width 6pt depth 0pt}

\def\edots{\mathinner{\mkern1mu
    \raise7pt\vbox{\kern7pt\hbox{.}}\mkern1mu   %  . shorter
    \raise4pt\hbox{.}\mkern1mu                  %     .
    \raise1pt\hbox{.}\mkern1mu}}                %        .
\def\fdots{\mathinner{\mkern1mu
    \raise7pt\vbox{\kern7pt\hbox{.}}            %   . ~45°
    \raise4pt\hbox{.}                           %     .
    \raise1pt\hbox{.}\mkern1mu}}                %       .

\def\mod#1{\allowbreak \mkern 10mu({\rm mod}\,\,#1)}
% redefines TeX's one using less space

%def\int{\intop\displaylimits}
%def\oint{\ointop\displaylimits}

%def\intoi{\int_0^1}
%def\intall{\int_{-\infty}^\infty}

%def\su#1{\mathop{\sum\raise0.7pt\hbox{$\S\!\!\!\!\!#1\,$}}}

%let\frakR\Re
%let\frakI\Im
%def\Re{\mathop{\rm Re}\nolimits}
%def\Im{\mathop{\rm Im}\nolimits}
%def\conj#1{\overline{#1\vphantom1}}
%def\cj#1{\overline{#1\vphantom+}}

%def\forAll{\mathop\forall\limits}
%def\Exists{\mathop\exists\limits}

%%%% PICTURES %%%%%%%%%%%%%%%%%%%%%%%%%%%%%%%%%%%%%%%%%%%

%def\cent{\makebox(0,0)}

%def\node{\circle*4}
%def\nOde{\circle4}

%%%% REFERENCES %%%%%%%%%%%%%%%%%%%%%%%%%%%%%%%%%%%%%%%%%

\def\opcit{[{\it op.\,cit.}]}
\def\bitem#1{\bibitem[#1]{#1}}
\def\name#1{{\sc #1}}
\def\book#1{{\sl #1\/}}
\def\paper#1{``#1''}
\def\mag#1{{\it #1\/}}
\def\vol#1{{\bf #1}}
\def\isbn#1{\hbox{\small\tt ISBN\,\,#1}}
\def\seq#1{{\small\tt #1}}
%def\url<{\verb>}

\def\@cite#1#2{[{#1\if@tempswa\ #2\fi}]}

%%%% VERBATIM CODE %%%%%%%%%%%%%%%%%%%%%%%%%%%%%%%%%%%%%%

\def\"{\verb"}

%%%% WORDS %%%%%%%%%%%%%%%%%%%%%%%%%%%%%%%%%%%%%%%%%%%%%%

% \hyphenation{pre-sent pre-sents pre-sent-ed pre-sent-ing
% re-pre-sent re-pre-sents re-pre-sent-ed re-pre-sent-ing
% re-fer-ence re-fer-ences re-fer-enced re-fer-encing
% ge-o-met-ry re-la-ti-vi-ty Gauss-ian Gauss-ians
% Des-ar-gues-ian}

%def\oord/{o{\trema o}rdin\-ate}
% usage: C\oord/s, c\oord/.
% output: co\"ord... except when linebreak at co-ord...

%%%%%%%%%%%%%%%%%%%%%%%%%%%%%%%%%%%%%%%%%%%%%%%%%%%%%%%%%

                                          %
                                          %          ^ ^
\catcode`\@=12                            %          ` '
                                          %         ->*<-
                                          %           ~
\begin{document}
\PMlinkescapeword{arcs}
\PMlinkescapeword{cycle}
\PMlinkescapeword{closed}
\PMlinkescapeword{length}
\PMlinkescapeword{node}
\PMlinkescapeword{nodes}
\PMlinkescapeword{open}
\PMlinkescapeword{order}
\PMlinkescapeword{sequence}
\PMlinkescapeword{term}
\PMlinkescapeword{terms}
\PMlinkescapeword{types}

Graph theory terminology is notoriously variable so the following definitions should be used with caution. In books, most authors define their usage at the beginning.

In a graph, multigraph or even pseudograph $G$,
%
\begin{itemize}

\item a {\bf walk} of length $s$ is formed by a sequence of $s$ edges such
      that any two successive edges in the sequence share a vertex (aka node).
      The walk is also considered to include all the vertices (nodes) incident
      to those edges, making it a subgraph.

      In the case of a simple graph (i.e.\ not a multigraph) it is also
      possible to define the walk uniquely by the vertices it visits: a
      {\bf walk} of length $s$ is then a sequence of vertices $\nu_0$, 
      $\nu_1$, \dots\ $\nu_s$ such that an edge $\nu_i\nu_{i+1}$ exists
      for all $0\le i\lt s$. Again the walk is considered to contain those
      edges as well as the vertices.

\item A {\bf trek} is a walk that does not backtrack, i.e.\ 
      no two successive edges are the same.

      For simple graphs this also implies
      $\nu_i \ne \nu_{i+2}$ for all $0\le i\le s-2$.

\item A {\bf trail} is a walk where all edges are distinct, and

\item a {\bf path} is one where all vertices are distinct.

\end{itemize}
%
The walk, etc.\ is said to {\bf run from $\nu_0$ to $\nu_s$}, to {\bf run between} them, to {\bf connect} them etc. The term {\em trek\/} was introduced by Cameron~\cite{Cam94} who notes the lexicographic mnemonic
$$
  \hbox{\em paths\/}  \;\subset\;
  \hbox{\em trails\/} \;\subset\;
  \hbox{\em treks\/}  \;\subset\;
  \hbox{\em walks\/}
$$
The other terms are fairly widespread, cf.~\cite{Wil02}, but {\bf beware:} many authors call walks {\bf paths}, and some then call paths {\bf chains}. And when edges are called {\bf arcs}, a trek of length $s$ sometimes goes by the name
{\bf $s$-arc}.

Note that for the purpose of defining connectivity any of these types of wanderings can be used; if a walk exists between vertices $\mu$ and $\nu$ then there also exists a path between them. And here we must allow $s=0$ to make ``are connected by a path'' an equivalence relation on vertices (in order to define
connected components as its equivalence classes).
%
\begin{itemize}

\item A {\bf closed walk} aka {\bf circuit} of length $s\ne0$ is a walk where
      $\nu_0 = \nu_s$, 

\item a {\bf closed trek} is a trek that's closed in the same way, and 

\item a {\bf closed trail} likewise;

\item a {\bf closed path} aka ({\bf elementary}) {\bf cycle} is like a path
      (except that we only demand that $\nu_i$ for $0\le i\lt s$ are distinct)
      and again closed ($\nu_s$ again coincides with $\nu_0$).

\end{itemize}
%
{\bf Beware:} cycles are often called circuits~\cite{Cam94}; the distinction between circuits and cycles here follows Wilson~\cite{Wil02}. These closed wanderings are often called after their length: {\bf $s$-circuits}, {\bf $s$-cycles}.

The case $s=0$ is excluded from these definitions; $1$-cycles are loops so imply a pseudograph; $2$-cycles are double edges implying multigraphs; so $3$ is the minimum cycle length in a proper graph.

Note also that in trivalent aka cubic graphs a closed trail is automatically a closed path: it is impossible to visit a vertex (in via edge $a$, out via edge $b$ say) and visit it again (in via $c$, out via $d$) without also revisiting an edge, because there are only three edges at each vertex.
%
\begin{itemize}

\item An {\bf open walk}, {\bf open trek}, {\bf open trail} is one that isn't
      closed.

\item An {\bf open path} (sometimes {\bf open chain}) is just a path as
      defined above (because a closed path isn't actually a path). Still,
      the term is useful when you want to emphasise the contrast with a
      closed path.

\end{itemize}

\begin{thebibliography}{MST73}

\bitem{Cam94} \name{Peter~J.~Cameron},
              \book{Combinatorics: topics, techniques, algorithms}\\
              Camb.~Univ.~Pr.~1994, \isbn{0\,521\,45761\,0},\\
{\tt\PMlinkexternal{http://www.maths.qmul.ac.uk/~pjc/comb/}{http://www.maths.qmul.ac.uk/~pjc/comb/}} 
              (solutions, errata \&c.)

\bitem{Wil02} \name{Robert~A.~Wilson},
              \book{Graphs, Colourings and the Four-colour Theorem},\\
              Oxford~Univ.~Pr.~2002, \isbn{0\,19\,851062\,4} (pbk),\\
{\tt\PMlinkexternal{http://www.maths.qmul.ac.uk/~raw/graph.html}{http://www.maths.qmul.ac.uk/~raw/graph.html}}
              (errata \&c.)

\end{thebibliography}
%%%%%
%%%%%
\end{document}
