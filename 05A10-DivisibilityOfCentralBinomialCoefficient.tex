\documentclass[12pt]{article}
\usepackage{pmmeta}
\pmcanonicalname{DivisibilityOfCentralBinomialCoefficient}
\pmcreated{2013-03-22 17:41:22}
\pmmodified{2013-03-22 17:41:22}
\pmowner{rspuzio}{6075}
\pmmodifier{rspuzio}{6075}
\pmtitle{divisibility of central binomial coefficient}
\pmrecord{8}{40130}
\pmprivacy{1}
\pmauthor{rspuzio}{6075}
\pmtype{Proof}
\pmcomment{trigger rebuild}
\pmclassification{msc}{05A10}
\pmclassification{msc}{11B65}

\endmetadata

% this is the default PlanetMath preamble.  as your knowledge
% of TeX increases, you will probably want to edit this, but
% it should be fine as is for beginners.

% almost certainly you want these
\usepackage{amssymb}
\usepackage{amsmath}
\usepackage{amsfonts}

% used for TeXing text within eps files
%\usepackage{psfrag}
% need this for including graphics (\includegraphics)
%\usepackage{graphicx}
% for neatly defining theorems and propositions
\usepackage{amsthm}
% making logically defined graphics
%%%\usepackage{xypic}

% there are many more packages, add them here as you need them

% define commands here
\newtheorem{thm}{Theorem}
\begin{document}
In this entry, we shall prove two results about the
divisibility of central binomial coefficients
which were stated in the main entry.

\begin{thm}
If $n \ge 3$ is an integer and $p$ is a prime number such that $n < p < 2n$, then
$p$ divides ${2n \choose n}$.
\end{thm}

\begin{proof}
We will examine the following expression for our binomial coefficient:
\[
{2n \choose n} =
{2n (2n-1) \cdots (n+2) (n+1) \over
n (n-1) \cdots 3 \cdot 2 \cdot 1}.
\]
Since $n < p < 2n$, we find $p$ appearing in the numerator.  However,
$p$ cannot appear in the denominator because the terms there are all
smaller than $n$.  Hence, $p$ cannot be cancelled, so it must divide
${2n \choose n}$.
\end{proof}

\begin{thm}
If $n \ge 3$ is an integer and $p$ is a prime number such that $2n/3 < p \le n$, then
$p$ does not divide ${2n \choose n}$.
\end{thm}

\begin{proof}
We will again examine our expression for our binomial coefficient:
\[
{2n \choose n} =
{2n (2n-1) \cdots (n+2) (n+1) \over
n (n-1) \cdots 3 \cdot 2 \cdot 1}.
\]
This time, because $2n/3 < p \le n$, we find $p$ appearing in the denominator
and $2p$ appearing in the numerator.  No other multiples will appear because,
if $m > 2$, then $mp > 2n$.  The two occurrences of $p$ noted above cancel, hence
$p$ is not a prime factor of ${2n \choose n}$.
\end{proof}
%%%%%
%%%%%
\end{document}
