\documentclass[12pt]{article}
\usepackage{pmmeta}
\pmcanonicalname{Subdivision}
\pmcreated{2013-03-22 12:31:51}
\pmmodified{2013-03-22 12:31:51}
\pmowner{CWoo}{3771}
\pmmodifier{CWoo}{3771}
\pmtitle{subdivision}
\pmrecord{5}{32772}
\pmprivacy{1}
\pmauthor{CWoo}{3771}
\pmtype{Definition}
\pmcomment{trigger rebuild}
\pmclassification{msc}{05C99}
\pmsynonym{topological minor}{Subdivision}
\pmrelated{Homeomorphic}
\pmrelated{Realization}

\endmetadata

% this is the default PlanetMath preamble.  as your knowledge
% of TeX increases, you will probably want to edit this, but
% it should be fine as is for beginners.

% almost certainly you want these
\usepackage{amssymb}
\usepackage{amsmath}
\usepackage{amsfonts}

% used for TeXing text within eps files
%\usepackage{psfrag}
% need this for including graphics (\includegraphics)
%\usepackage{graphicx}
% for neatly defining theorems and propositions
%\usepackage{amsthm}
% making logically defined graphics
%%%\usepackage{xypic} 

% there are many more packages, add them here as you need them

% define commands here
\begin{document}
A graph $H$ is said to be a \emph{subdivision}, or \emph{topological minor} of a graph $G$, or a \emph{topological $G$ graph} if $H$ is obtained from $G$ by subdividing some of the edges, that is, by replacing the edges by paths having at most their endvertices in common. We often use $TG$ for a topological $G$ graph.

Thus, $TG$ denotes \emph{any} member of a large family of graphs; for example, $TC_4$ is an arbitrary cycle of length at least 4. For any graph $G$, the spaces $R(G)$ (denoting the realization of G) and $R(TG)$ are homeomorphic.


\footnotesize{Adapted with permission of the author from \emph{\PMlinkescapetext{Modern Graph Theory}} by B\'{e}la Bollob\'{a}s, published by Springer-Verlag New York, Inc., 1998.}
%%%%%
%%%%%
\end{document}
