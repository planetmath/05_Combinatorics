\documentclass[12pt]{article}
\usepackage{pmmeta}
\pmcanonicalname{Matching}
\pmcreated{2013-03-22 12:40:00}
\pmmodified{2013-03-22 12:40:00}
\pmowner{Mathprof}{13753}
\pmmodifier{Mathprof}{13753}
\pmtitle{matching}
\pmrecord{7}{32939}
\pmprivacy{1}
\pmauthor{Mathprof}{13753}
\pmtype{Definition}
\pmcomment{trigger rebuild}
\pmclassification{msc}{05C70}
\pmrelated{MaximalMatchingminimalEdgeCoveringTheorem}
\pmrelated{Matching}
\pmrelated{EdgeCovering}
\pmrelated{Saturate}
\pmdefines{maximal matching}
\pmdefines{perfect matching}

% this is the default PlanetMath preamble.  as your knowledge
% of TeX increases, you will probably want to edit this, but
% it should be fine as is for beginners.

% almost certainly you want these
\usepackage{amssymb}
\usepackage{amsmath}
\usepackage{amsfonts}

% used for TeXing text within eps files
%\usepackage{psfrag}
% need this for including graphics (\includegraphics)
%\usepackage{graphicx}
% for neatly defining theorems and propositions
%\usepackage{amsthm}
% making logically defined graphics
%%%\usepackage{xypic} 

% there are many more packages, add them here as you need them

% define commands here
\begin{document}
Let $G$ be a graph.  A \emph{matching} $M$ on $G$ is a subset of the edges of $G$ such that each vertex in $G$ is incident with no more than one edge in $M$.

It is easy to find a matching on a graph; for example, the empty set will always be a matching.  Typically, the most interesting matchings are \emph{maximal matchings}.  A maximal matching on a graph $G$ is simply a matching of the largest possible size.

A \emph{perfect matching} is a matching that saturates every vertex.

%%%%%
%%%%%
\end{document}
