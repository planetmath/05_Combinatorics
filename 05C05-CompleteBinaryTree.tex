\documentclass[12pt]{article}
\usepackage{pmmeta}
\pmcanonicalname{CompleteBinaryTree}
\pmcreated{2013-03-22 12:30:14}
\pmmodified{2013-03-22 12:30:14}
\pmowner{akrowne}{2}
\pmmodifier{akrowne}{2}
\pmtitle{complete binary tree}
\pmrecord{7}{32736}
\pmprivacy{1}
\pmauthor{akrowne}{2}
\pmtype{Definition}
\pmcomment{trigger rebuild}
\pmclassification{msc}{05C05}
\pmsynonym{complete}{CompleteBinaryTree}
\pmrelated{ExtendedBinaryTree}

\usepackage{amssymb}
\usepackage{amsmath}
\usepackage{amsfonts}

%\usepackage{psfrag}
%\usepackage{graphicx}
%%%\usepackage{xypic}
\begin{document}
A \emph{complete binary tree} is a binary tree with the additional property that every node must have exactly two ``children'' if an internal node, and zero children if a leaf node.  

More precisely: for our base case, the complete binary tree of exactly one node is simply the tree consisting of that node by itself.  The property of being ``complete'' is preserved if, at each step, we expand the tree by connecting exactly zero or two individual nodes (or complete binary trees) to any node in the tree (but both must be connected to the same node.)
%%%%%
%%%%%
\end{document}
