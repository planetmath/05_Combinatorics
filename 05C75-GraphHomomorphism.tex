\documentclass[12pt]{article}
\usepackage{pmmeta}
\pmcanonicalname{GraphHomomorphism}
\pmcreated{2013-03-22 16:04:53}
\pmmodified{2013-03-22 16:04:53}
\pmowner{CWoo}{3771}
\pmmodifier{CWoo}{3771}
\pmtitle{graph homomorphism}
\pmrecord{11}{38140}
\pmprivacy{1}
\pmauthor{CWoo}{3771}
\pmtype{Definition}
\pmcomment{trigger rebuild}
\pmclassification{msc}{05C75}
\pmclassification{msc}{05C60}
\pmsynonym{graph automorphism}{GraphHomomorphism}
\pmrelated{GraphIsomorphism}
\pmrelated{Pseudograph}
\pmrelated{Multigraph}
\pmrelated{Graph}

\endmetadata

\usepackage{amssymb,amscd}
\usepackage{amsmath}
\usepackage{amsfonts}

% used for TeXing text within eps files
%\usepackage{psfrag}
% need this for including graphics (\includegraphics)
%\usepackage{graphicx}
% for neatly defining theorems and propositions
%\usepackage{amsthm}
% making logically defined graphics
%%\usepackage{xypic}
\usepackage{pst-plot}
\usepackage{psfrag}

% define commands here

\begin{document}
We shall define what a graph homomorphism is between two pseudographs, so that a graph homomorphism between two multigraphs, or two graphs are special cases.

Recall that a pseudograph $G$ is an ordered triple $(V,E,i)$, where $V$ is a set called the vertex set of $G$, $E$ is a set called the edge set of $G$, and $i:E\to 2^V$ is a map (incidence map), such that for every $e\in E$, $1\le |i(e)|\le 2$.  

Elements of $V$ are called vertices, elements of $E$ edges.  The vertex/vertices associated with any edge $e$ via the map $i$ is/are called the endpoint(s) of $e$.  If an edge $e$ has only one endpoint, it is called a loop.  $e_1$ and $e_2$ are said to be parallel if $i(e_1)=i(e_2)$.  

As two examples, a multigraph is a pseudograph such that $|i(e)|=2$ for each edge $e\in E$ (no loops allowed).  A graph is a multigraph such that $i$ is one-to-one (no parallel edges allowed).

\textbf{Definition}.  Given two pseudographs $G_1=(V_1,E_1,i_1)$ and $G_2=(V_2,E_2,i_2)$, a \emph{graph homomorphism} $h$ from $G_1$ to $G_2$ consists of two functions $f:V_1\to V_2$ and $g:E_1\to E_2$, such that 
\begin{eqnarray}
i_2\circ g=f^*\circ i_1.
\end{eqnarray}
The function $f^*:2^{V_1} \to 2^{V_2}$ is defined as $f^*(S)=\lbrace f(s)\mid s\in S\rbrace$.

A \emph{graph isomorphism} $h=(f,g)$ is a graph homomorphism such that both $f$ and $g$ are bijections.  It is a 
\emph{graph automorphism} if $G_1=G_2$.

\textbf{Remarks}.  
\begin{itemize}
\item
In case when $G_1$ and $G_2$ are graphs, a graph homomorphism may be defined in terms of a single function $f:V_1\to V_2$ satisfying the condition (*)
\begin{center}
$\lbrace v_1,v_2\rbrace$ is an edge of $G_1\Longrightarrow \lbrace f(v_1),f(v_2)\rbrace$ is an edge of $G_2$.  
\end{center}
To see this, suppose $h=(f,g)$ is a graph homomorphism from $G_1$ to $G_2$.  Then $f^*i_1(e)=f^*(\lbrace v_1,v_2\rbrace)=\lbrace f(v_1),f(v_2)\rbrace=i_2(g(e))$.  The last equation says that $f(v_1)$ and $f(v_2)$ are endpoints of $g(e)$.  Conversely, suppose $f:V_1\to V_2$ is a function sastisfying condition (*).  For each $e\in E_1$ with end points $\lbrace v_1,v_2\rbrace$, let $e^{\prime}\in E_2$ be the edge whose endpoints are $\lbrace f(v_1),f(v_2)\rbrace$.  There is only one such edge because $G_2$ is a graph, having no parallel edges.  Define $g:E_1\to E_2$ by $g(e)=e^{\prime}$.  Then $g$ is a well-defined function which also satisfies Equation (1).  So $h:=(f,g)$ is a graph homomorphism $G_1\to G_2$.
\item
The definition of a graph homomorphism between pseudographs can be analogously applied to one between directed pseudographs.  Since the incidence map $i$ now maps each edge to an ordered pair of vertices, a graph homomorphism thus defined will respect the ``direction'' of each edge.  For example, 
\[
\xymatrix{a \ar[r] & b\ar[d] \\ 
c \ar[u] & d\ar[l]}
\qquad\qquad\qquad\qquad
\xymatrix{r \ar[r] & s \\ 
t \ar[u]\ar[r] & u\ar[u]}
\]
are two non-isomorphic digraphs whose underlying graphs are isomorphic.  Note that one digraph is strongly connected, while the other is only weakly connected.
\end{itemize}
%%%%%
%%%%%
\end{document}
