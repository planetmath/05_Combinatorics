\documentclass[12pt]{article}
\usepackage{pmmeta}
\pmcanonicalname{PrimePartition}
\pmcreated{2013-03-22 17:28:02}
\pmmodified{2013-03-22 17:28:02}
\pmowner{PrimeFan}{13766}
\pmmodifier{PrimeFan}{13766}
\pmtitle{prime partition}
\pmrecord{4}{39852}
\pmprivacy{1}
\pmauthor{PrimeFan}{13766}
\pmtype{Definition}
\pmcomment{trigger rebuild}
\pmclassification{msc}{05A17}
\pmclassification{msc}{11P99}

\endmetadata

% this is the default PlanetMath preamble.  as your knowledge
% of TeX increases, you will probably want to edit this, but
% it should be fine as is for beginners.

% almost certainly you want these
\usepackage{amssymb}
\usepackage{amsmath}
\usepackage{amsfonts}

% used for TeXing text within eps files
%\usepackage{psfrag}
% need this for including graphics (\includegraphics)
%\usepackage{graphicx}
% for neatly defining theorems and propositions
%\usepackage{amsthm}
% making logically defined graphics
%%%\usepackage{xypic}

% there are many more packages, add them here as you need them

% define commands here

\begin{document}
A {\em prime partition} is a \PMlinkname{partition}{IntegerPartition} of a given positive integer $n$ consisting only of prime numbers. For example, a prime partition of 42 is 29 + 5 + 5 + 3.

If we accept partitions of length 1 as valid partitions, then it is obvious that only prime numbers have prime partitions of length 1. Not accepting 1 as a prime number makes the problem of prime partitions more interesting, otherwise there would always be for a given $n$, if nothing else, a prime partition consisting of $n$ 1s. Almost as bad, however, is a partion of $n$ into $\lfloor \frac{n}{2} \rfloor$ 2s and 3s.

Both Goldbach's conjecture and Levy's conjecture can be restated in terms of prime partitions thus: for any even integer $n > 2$ there is always a prime partition of length 2, and for any odd integer $n > 5$ there is always a prime partition of length 3 with at most 2 distinct elements.

Assuming Goldbach's conjecture is true, the most efficient prime partition of an even integer is of length 2, while Vinogradov's theorem has proven the most efficient prime partition of a sufficiently large composite odd integer is of length 3.
%%%%%
%%%%%
\end{document}
