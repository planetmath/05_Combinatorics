\documentclass[12pt]{article}
\usepackage{pmmeta}
\pmcanonicalname{GeneratingFunctionForTheReciprocalCatalanNumbers}
\pmcreated{2013-03-22 19:05:12}
\pmmodified{2013-03-22 19:05:12}
\pmowner{juanman}{12619}
\pmmodifier{juanman}{12619}
\pmtitle{generating function for the reciprocal Catalan numbers}
\pmrecord{10}{41975}
\pmprivacy{1}
\pmauthor{juanman}{12619}
\pmtype{Derivation}
\pmcomment{trigger rebuild}
\pmclassification{msc}{05A19}
\pmclassification{msc}{05A15}
\pmclassification{msc}{05A10}
%\pmkeywords{sum of reciprocals}
%\pmkeywords{generating function}
\pmrelated{CatalanNumbers}

\endmetadata

% this is the default PlanetMath preamble.  as your knowledge
% of TeX increases, you will probably want to edit this, but
% it should be fine as is for beginners.

% almost certainly you want these
\usepackage{amssymb}
\usepackage{amsmath}
\usepackage{amsfonts}

% used for TeXing text within eps files
\usepackage{psfrag}
% need this for including graphics (\includegraphics)
\usepackage{graphicx}
% for neatly defining theorems and propositions
%\usepackage{amsthm}
% making logically defined graphics
%%\usepackage{xypic}

% there are many more packages, add them here as you need them

% define commands here

\begin{document}
The series 
$$1+x+\frac{x^2}{2}+\frac{x^3}{5}+\frac{x^4}{14}+\frac{x^5}{42}+\frac{x^6}{132}+\frac{x^7}{429}+\cdots$$ 
whose coefficients are the reciprocal of the Catalan numbers $\frac{{2n\choose n}}{n+1}$, has as a generating function $$\frac{2\,\left( {\sqrt{4-x}}\,\left(8+x\right)+12\,{\sqrt{x}}\,\arctan (\frac{{\sqrt{x}}}{{\sqrt{4 - x}}}) \right) }{\sqrt{\left( 4 - x \right)^5}}$$

To deduce such a formula the easy way, one starts from the generating function of the reciprocal central binomial coefficients and having into account the relation
$$\frac{d}{dx}\left(\frac{x^{n+1}}{{2n\choose n}}\right)=\frac{(n+1)x^n}{{2n\choose n}}$$
for each term in the corresponding series and applied to the function in the region of uniform convergence.
Another method is almost exaclty the same like in the derivation of the generating function for the reciprocal central binomial coefficients.
%%%%%
%%%%%
\end{document}
