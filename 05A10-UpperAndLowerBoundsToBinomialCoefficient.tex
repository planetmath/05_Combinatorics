\documentclass[12pt]{article}
\usepackage{pmmeta}
\pmcanonicalname{UpperAndLowerBoundsToBinomialCoefficient}
\pmcreated{2013-03-22 13:29:53}
\pmmodified{2013-03-22 13:29:53}
\pmowner{rspuzio}{6075}
\pmmodifier{rspuzio}{6075}
\pmtitle{upper and lower bounds to binomial coefficient}
\pmrecord{6}{34074}
\pmprivacy{1}
\pmauthor{rspuzio}{6075}
\pmtype{Theorem}
\pmcomment{trigger rebuild}
\pmclassification{msc}{05A10}

\endmetadata

% this is the default PlanetMath preamble.  as your knowledge
% of TeX increases, you will probably want to edit this, but
% it should be fine as is for beginners.

% almost certainly you want these
\usepackage{amssymb}
\usepackage{amsmath}
\usepackage{amsfonts}

% used for TeXing text within eps files
%\usepackage{psfrag}
% need this for including graphics (\includegraphics)
%\usepackage{graphicx}
% for neatly defining theorems and propositions
%\usepackage{amsthm}
% making logically defined graphics
%%%\usepackage{xypic}

% there are many more packages, add them here as you need them

% define commands here
\begin{document}
Given two integers $n,k>0$ such that $k\le n$, we have the following inequalities for the binomial coefficient ${n\choose k}$:
\begin{eqnarray*}
{n \choose k} & \le & \frac{n^k}{k!} \\
{n \choose k} & \le & \left(\frac{n\cdot e}{k}\right)^k \\
{n \choose k} & \ge & \left(\frac{n}{k}\right)^k \\
\end{eqnarray*}
Here $e$ is the base of natural logarithms.
Also, for large $n$, ${n \choose k} \approx \frac{n^k}{k!}$.
%%%%%
%%%%%
\end{document}
