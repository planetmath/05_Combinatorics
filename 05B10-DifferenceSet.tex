\documentclass[12pt]{article}
\usepackage{pmmeta}
\pmcanonicalname{DifferenceSet}
\pmcreated{2013-03-22 16:50:04}
\pmmodified{2013-03-22 16:50:04}
\pmowner{CWoo}{3771}
\pmmodifier{CWoo}{3771}
\pmtitle{difference set}
\pmrecord{9}{39077}
\pmprivacy{1}
\pmauthor{CWoo}{3771}
\pmtype{Definition}
\pmcomment{trigger rebuild}
\pmclassification{msc}{05B10}
\pmdefines{non-trivial difference set}
\pmdefines{planar difference set}

\usepackage{amssymb,amscd}
\usepackage{amsmath}
\usepackage{amsfonts}

% used for TeXing text within eps files
%\usepackage{psfrag}
% need this for including graphics (\includegraphics)
%\usepackage{graphicx}
% for neatly defining theorems and propositions
\usepackage{amsthm}
% making logically defined graphics
%%\usepackage{xypic}
\usepackage{pst-plot}
\usepackage{psfrag}

% define commands here
\newtheorem{prop}{Proposition}
\newtheorem{thm}{Theorem}
\newtheorem{ex}{Example}
\newcommand{\real}{\mathbb{R}}
\begin{document}
\textbf{Definition}.  Let $A$ be a finite abelian group of order $n$.  A subset $D$ of $A$ is said to be a \emph{difference set} (in $A$) if there is a positive integer $m$ such that every non-zero element of $A$ can be expressed as the difference of elements of $D$ in exactly $m$ ways.  

If $D$ has $d$ elements, then we have the equation $$m(n-1)=d(d-1).$$  In the equation, we are counting the number of pairs of distinct elements of $D$.  On the left hand side, we are counting it by noting that there are $m(n-1)$ pairs of elements of $D$ such that their difference is non-zero.  On the right hand side, we first count the number of elements in $D^2$, which is $d^2$, then subtracted by $d$, since there are $d$ pairs of $(x,y)\in D^2$ such that $x=y$.

A difference set with parameters $n,m,d$ defined above is also called a $(n,d,m)$-difference set.  A difference set is said to be \emph{non-trivial} if $1<d <n-1$.  A difference set is said to be \emph{planar} if $m=1$.

\textbf{Difference sets versus square designs}.  Recall that a square design is a $\tau$-$(\nu,\kappa,\lambda)$-\PMlinkname{design}{Design} where $\tau=2$ and the number $\nu$ of points is the same as the number $b$ of blocks.  In a general design, $b$ is related to the other numbers by the equation $$b\binom{\kappa}{\tau}=\lambda\binom{\nu}{\tau}.$$  So in a square design, the equation reduces to $b\kappa(\kappa-1)=\lambda\nu(\nu-1)$, or $$\lambda(\nu-1)=\kappa(\kappa-1),$$
which is identical to the equation above for the difference set.  A square design with parameters $\lambda,\nu,\kappa$ is called a square $(\nu,\kappa,\lambda)$-design.  

One can show that a subset $D$ of an abelian group $A$ is an $(n,d,m)$-difference set iff it is a square $(n,d,m)$-design where $A$ is the set of points and $\lbrace D+a\mid a\in A\rbrace$ is the set of blocks.
%%%%%
%%%%%
\end{document}
