\documentclass[12pt]{article}
\usepackage{pmmeta}
\pmcanonicalname{MagicConstant}
\pmcreated{2013-03-22 16:24:57}
\pmmodified{2013-03-22 16:24:57}
\pmowner{PrimeFan}{13766}
\pmmodifier{PrimeFan}{13766}
\pmtitle{magic constant}
\pmrecord{5}{38565}
\pmprivacy{1}
\pmauthor{PrimeFan}{13766}
\pmtype{Definition}
\pmcomment{trigger rebuild}
\pmclassification{msc}{05B15}

% this is the default PlanetMath preamble.  as your knowledge
% of TeX increases, you will probably want to edit this, but
% it should be fine as is for beginners.

% almost certainly you want these
\usepackage{amssymb}
\usepackage{amsmath}
\usepackage{amsfonts}

% used for TeXing text within eps files
%\usepackage{psfrag}
% need this for including graphics (\includegraphics)
%\usepackage{graphicx}
% for neatly defining theorems and propositions
%\usepackage{amsthm}
% making logically defined graphics
%%%\usepackage{xypic}

% there are many more packages, add them here as you need them

% define commands here

\begin{document}
Given a magic square, magic cube, etc., the sum of any row, column or diagonal is called the \emph{magic constant} of that magic square, cube, etc.

In the case of a standard $n \times n$ magic square that uses the integers from 1 to $n^2$, the magic constant is $$\frac{1}{n}\sum_{i = 1}^{n^2} i,$$ while that for a magic cube is $$\frac{1}{n^2}\sum_{i = 1}^{n^3} i.$$ We can then generalize to higher dimensions $d$ thus: $$\frac{1}{n^{d - 1}}\sum_{i = 1}^{n^d} i.$$

So, for dimension $d$ the magic constant is $\frac{n(n^d + 1)}{2}$. For instance, a Franklin magic square ($n = 8,d = 2$) has magic constant $\frac{8(8^2 + 1)}{2} = 260$.

In a trivial sense, an $n \times n$ sudoku puzzle has a magic constant of $n^2$.
%%%%%
%%%%%
\end{document}
