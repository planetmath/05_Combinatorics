\documentclass[12pt]{article}
\usepackage{pmmeta}
\pmcanonicalname{GeneratingFunctionForTheReciprocalCentralBinomialCoefficients}
\pmcreated{2013-03-22 18:58:09}
\pmmodified{2013-03-22 18:58:09}
\pmowner{juanman}{12619}
\pmmodifier{juanman}{12619}
\pmtitle{generating function for  the reciprocal central binomial coefficients}
\pmrecord{12}{41829}
\pmprivacy{1}
\pmauthor{juanman}{12619}
\pmtype{Result}
\pmcomment{trigger rebuild}
\pmclassification{msc}{05A19}
\pmclassification{msc}{11B65}
\pmclassification{msc}{05A10}
\pmclassification{msc}{05A15}
\pmsynonym{convergent series}{GeneratingFunctionForTheReciprocalCentralBinomialCoefficients}
%\pmkeywords{sum of reciprocals}
%\pmkeywords{generating function}

\endmetadata

% this is the default PlanetMath preamble.  as your knowledge
% of TeX increases, you will probably want to edit this, but
% it should be fine as is for beginners.

% almost certainly you want these
\usepackage{amssymb}
\usepackage{amsmath}
\usepackage{amsfonts}

% used for TeXing text within eps files
%\usepackage{psfrag}
% need this for including graphics (\includegraphics)
%\usepackage{graphicx}
% for neatly defining theorems and propositions
%\usepackage{amsthm}
% making logically defined graphics
%%%\usepackage{xypic}

% there are many more packages, add them here as you need them

% define commands here

\begin{document}
It is well known that the sequence called central binomial coefficients is defined by ${2n\choose n}$ and whose initial terms are $1,2,6,20,70, 252,...$ has a generating function $\frac{1}{\sqrt{1-4x}}$.
But it is less known the fact that
the function
$$\frac{4\,\left( {\sqrt{4 - x}} + {\sqrt{x}}\,\arcsin (\frac{{\sqrt{x}}}{2}) \right) }{{\sqrt{(4 - x)^3}}}
$$
has ordinary power series
$$1+\frac{x}{2}+\frac{x^2}{6}+\frac{x^3}{20}+\frac{x^4}{70}+\frac{x^5}{252}+...$$
This means that such a function is a generating function for the reciprocals ${2n\choose n}^{-1}$.

From that expression we can see that the numerical series $\sum_{n=0}^{\infty}{2n\choose n}^{-1}$ sums $\frac{4\,\left( {\sqrt{3}} + \frac{\pi }{6} \right) }{3\,{\sqrt{3}}}$ which has the approximate value $1,\!7363998587187151$.

\vskip1cm

Reference:

1) Renzo Sprugnoli, {\it Sum of reciprocals of the Central Binomial Coefficients}, Integers: electronic 
journal of combinatorial number theory, 6 (2006) A27, 1-18
%%%%%
%%%%%
\end{document}
