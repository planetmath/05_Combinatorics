\documentclass[12pt]{article}
\usepackage{pmmeta}
\pmcanonicalname{PropertyB}
\pmcreated{2013-03-22 13:39:10}
\pmmodified{2013-03-22 13:39:10}
\pmowner{bbukh}{348}
\pmmodifier{bbukh}{348}
\pmtitle{property B}
\pmrecord{6}{34306}
\pmprivacy{1}
\pmauthor{bbukh}{348}
\pmtype{Definition}
\pmcomment{trigger rebuild}
\pmclassification{msc}{05C15}

\usepackage{amssymb}
\usepackage{amsmath}
\usepackage{amsfonts}

\makeatletter
\@ifundefined{bibname}{}{\renewcommand{\bibname}{References}}
\makeatother
\begin{document}
A hypergraph $G$ is said to possess \emph{property B} if it $2$-colorable, i.e., its vertices can be colored in two colors, so that no edge of $G$ is monochromatic.

The property was named after Felix Bernstein by E. W. Miller.
%%%%%
%%%%%
\end{document}
