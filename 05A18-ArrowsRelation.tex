\documentclass[12pt]{article}
\usepackage{pmmeta}
\pmcanonicalname{ArrowsRelation}
\pmcreated{2013-03-22 17:48:54}
\pmmodified{2013-03-22 17:48:54}
\pmowner{Henry}{455}
\pmmodifier{Henry}{455}
\pmtitle{arrows relation}
\pmrecord{5}{40278}
\pmprivacy{1}
\pmauthor{Henry}{455}
\pmtype{Definition}
\pmcomment{trigger rebuild}
\pmclassification{msc}{05A18}
\pmclassification{msc}{03E05}
\pmrelated{PartitionsLessThanCofinality}
\pmrelated{ErdosRadoTheorem}
\pmdefines{homogeneous}
\pmdefines{arrows}
\pmdefines{homogeneous set}
\pmdefines{homogeneous subset}

\endmetadata

% this is the default PlanetMath preamble.  as your knowledge
% of TeX increases, you will probably want to edit this, but
% it should be fine as is for beginners.

% almost certainly you want these
\usepackage{amssymb}
\usepackage{amsmath}
\usepackage{amsfonts}

% used for TeXing text within eps files
%\usepackage{psfrag}
% need this for including graphics (\includegraphics)
%\usepackage{graphicx}
% for neatly defining theorems and propositions
%\usepackage{amsthm}
% making logically defined graphics
%%%\usepackage{xypic}

% there are many more packages, add them here as you need them

% define commands here
%\PMlinkescapeword{theory}
\begin{document}
Let $[X]^\alpha=\{Y\subseteq X\mid |Y|=\alpha\}$, that is, the set of subsets of $X$ of size $\alpha$. Then given some cardinals $\kappa$, $\lambda$, $\alpha$ and $\beta$

$$ \kappa\rightarrow(\lambda)^\alpha_\beta$$

states that for any set $X$ of size $\kappa$ and any function $f:[X]^\alpha\rightarrow\beta$, there is some $Y\subseteq X$ and some $\gamma\in\beta$ such that $|Y|=\lambda$ and for any $y\in [Y]^\alpha$, $f(y)=\gamma$.

In words, if $f$ is a partition of $[X]^\alpha$ into $\beta$ subsets then $f$ is constant on a subset of size $\lambda$ (a \emph{homogeneous} subset).

As an example, the pigeonhole principle is the statement that if $n$ is finite and $k<n$ then:

$$n\rightarrow 2^1_k$$

That is, if you try to partition $n$ into fewer than $n$ pieces then one piece has more than one element.

Observe that if

$$ \kappa\rightarrow(\lambda)^\alpha_\beta$$

then the same statement holds if:
\begin{itemize}

\item $\kappa$ is made larger (since the restriction of $f$ to a set of size $\kappa$ can be considered)

\item $\lambda$ is made smaller (since a subset of the homogeneous set will suffice)

\item $\beta$ is made smaller (since any partition into fewer than $\beta$ pieces can be expanded by adding empty sets to the partition)

\item $\alpha$ is made smaller (since a partition $f$ of $[\kappa]^\gamma$ where $\gamma<\alpha$ can be extended to a partition $f^\prime$ of $[\kappa]^\alpha$ by $f^\prime(X)=f(X_\gamma)$ where $X_\gamma$ is the $\gamma$ smallest elements of $X$)

\end{itemize}

$$\kappa\nrightarrow(\lambda)^\alpha_\beta$$

is used to state that the corresponding $\rightarrow$ relation is false.

\PMlinkescapeword{size}

{\bf References}
\begin{itemize}
\item Jech, T.  \emph{Set Theory}, Springer-Verlag, 2003
\item Just, W. and Weese, M. \emph{Topics in Discovering Modern Set Theory, II}, American Mathematical Society, 1996
\end{itemize}
%%%%%
%%%%%
\end{document}
