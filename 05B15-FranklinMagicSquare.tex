\documentclass[12pt]{article}
\usepackage{pmmeta}
\pmcanonicalname{FranklinMagicSquare}
\pmcreated{2013-03-22 16:24:52}
\pmmodified{2013-03-22 16:24:52}
\pmowner{PrimeFan}{13766}
\pmmodifier{PrimeFan}{13766}
\pmtitle{Franklin magic square}
\pmrecord{4}{38563}
\pmprivacy{1}
\pmauthor{PrimeFan}{13766}
\pmtype{Example}
\pmcomment{trigger rebuild}
\pmclassification{msc}{05B15}
\pmclassification{msc}{01A50}
\pmsynonym{Franklin square}{FranklinMagicSquare}

\endmetadata

% this is the default PlanetMath preamble.  as your knowledge
% of TeX increases, you will probably want to edit this, but
% it should be fine as is for beginners.

% almost certainly you want these
\usepackage{amssymb}
\usepackage{amsmath}
\usepackage{amsfonts}

% used for TeXing text within eps files
%\usepackage{psfrag}
% need this for including graphics (\includegraphics)
%\usepackage{graphicx}
% for neatly defining theorems and propositions
%\usepackage{amsthm}
% making logically defined graphics
%%%\usepackage{xypic}

% there are many more packages, add them here as you need them

% define commands here

\begin{document}
One day in 1771, Benjamin Franklin, tired of hearing political debates, amused himself by creating the following magic square, now called a {\em Franklin magic square}:

$$\begin{bmatrix}
52 & 61 & 4 & 13 & 20 & 29 & 36 & 45 \\
14 & 3 & 62 & 51 & 46 & 35 & 30 & 19 \\
53 & 60 & 5 & 12 & 21 & 28 & 37 & 44 \\
11 & 6 & 59 & 54 & 43 & 38 & 27 & 22 \\
55 & 58 & 7 & 10 & 23 & 26 & 39 & 42 \\
9 & 8 & 57 & 56 & 41 & 40 & 25 & 24 \\
50 & 63 & 2 & 15 & 18 & 31 & 34 & 47 \\
16 & 1 & 64 & 49 & 48 & 33 & 32 & 17 \\
\end{bmatrix}$$

The magic constant is 260. Furthermore, half any row or column (positions 1 to 4 or 5 to 8) equals half the magic constant. Two centuries later, Joseph Madachy realized that some half diagonals from the corner to the center also give 260.

Some other 8 by 8 magic squares with these properties are also called Franklin magic squares.
%%%%%
%%%%%
\end{document}
