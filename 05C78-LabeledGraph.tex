\documentclass[12pt]{article}
\usepackage{pmmeta}
\pmcanonicalname{LabeledGraph}
\pmcreated{2013-03-22 17:38:19}
\pmmodified{2013-03-22 17:38:19}
\pmowner{CWoo}{3771}
\pmmodifier{CWoo}{3771}
\pmtitle{labeled graph}
\pmrecord{7}{40060}
\pmprivacy{1}
\pmauthor{CWoo}{3771}
\pmtype{Definition}
\pmcomment{trigger rebuild}
\pmclassification{msc}{05C78}
\pmsynonym{labelled graph}{LabeledGraph}
\pmsynonym{graph labelling}{LabeledGraph}
\pmsynonym{labelling}{LabeledGraph}
\pmsynonym{vertex labelling}{LabeledGraph}
\pmsynonym{edge labelling}{LabeledGraph}
\pmsynonym{total labelling}{LabeledGraph}
\pmsynonym{labelled tree}{LabeledGraph}
\pmsynonym{labelled multigraph}{LabeledGraph}
\pmsynonym{labelled pseudograph}{LabeledGraph}
\pmdefines{graph labeling}
\pmdefines{labeling}
\pmdefines{vertex labeling}
\pmdefines{edge labeling}
\pmdefines{total labeling}
\pmdefines{labeled tree}
\pmdefines{labeled multigraph}
\pmdefines{labeled pseudograph}

\usepackage{amssymb,amscd}
\usepackage{amsmath}
\usepackage{amsfonts}
\usepackage{mathrsfs}

% used for TeXing text within eps files
%\usepackage{psfrag}
% need this for including graphics (\includegraphics)
%\usepackage{graphicx}
% for neatly defining theorems and propositions
\usepackage{amsthm}
% making logically defined graphics
%%\usepackage{xypic}
\usepackage{pst-plot}
\usepackage{psfrag}

% define commands here
\newcommand*{\abs}[1]{\left\lvert #1\right\rvert}
\newtheorem{prop}{Proposition}
\newtheorem{thm}{Theorem}
\newtheorem{ex}{Example}
\newcommand{\real}{\mathbb{R}}
\newcommand{\pdiff}[2]{\frac{\partial #1}{\partial #2}}
\newcommand{\mpdiff}[3]{\frac{\partial^#1 #2}{\partial #3^#1}}
\begin{document}
Let $G=(V,E)$ be a graph with vertex set $V$ and edge set $E$.  A \emph{labeling} of $G$ is a partial function $\ell: V\cup E\to L$ for some set $L$.  For every $x$ in the domain of $\ell$, the element $\ell(x)\in L$ is called the \emph{label} of $x$.  Three of the most common types of labelings of a graph $G$ are
\begin{itemize}
\item \emph{total labeling}: $\ell$ is a total function (defined for all of $V\cup E$),
\item \emph{vertex labeling}: the domain of $\ell$ is $V$, and
\item \emph{edge labeling}: the domain of $\ell$ is $E$.
\end{itemize}
Usually, $L$ above is assumed to be a set of integers.  A \emph{labeled graph} is a pair $(G,\ell)$ where $G$ is a graph and $\ell$ is a labeling of $G$.

An example of a labeling of a graph is a coloring of a graph.  Uses of graph labeling outside of combinatorics can be found in areas such as order theory, language theory, and proof theory.  A proof tree, for instance, is really a \emph{labeled tree}, where the labels of vertices are formulas, and the labels of edges are rules of inference.

\textbf{Remarks}.  
\begin{itemize}
\item
Every labeling of a graph can be extended to a total labeling.
\item
The notion of labeling can be easily extended to digraphs, multigraphs, and pseudographs.
\end{itemize}
%%%%%
%%%%%
\end{document}
