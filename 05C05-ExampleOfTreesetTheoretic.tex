\documentclass[12pt]{article}
\usepackage{pmmeta}
\pmcanonicalname{ExampleOfTreesetTheoretic}
\pmcreated{2013-03-22 12:52:27}
\pmmodified{2013-03-22 12:52:27}
\pmowner{uzeromay}{4983}
\pmmodifier{uzeromay}{4983}
\pmtitle{example of tree (set theoretic)}
\pmrecord{5}{33213}
\pmprivacy{1}
\pmauthor{uzeromay}{4983}
\pmtype{Example}
\pmcomment{trigger rebuild}
\pmclassification{msc}{05C05}
\pmclassification{msc}{03E05}
\pmrelated{CofinalBranch}

\endmetadata

% this is the default PlanetMath preamble.  as your knowledge
% of TeX increases, you will probably want to edit this, but
% it should be fine as is for beginners.

% almost certainly you want these
\usepackage{amssymb}
\usepackage{amsmath}
\usepackage{amsfonts}

% used for TeXing text within eps files
%\usepackage{psfrag}
% need this for including graphics (\includegraphics)
%\usepackage{graphicx}
% for neatly defining theorems and propositions
%\usepackage{amsthm}
% making logically defined graphics
%%%\usepackage{xypic}

% there are many more packages, add them here as you need them

% define commands here
%\PMlinkescapeword{theory}
\begin{document}
The set $\mathbb{Z}^+$ is a tree with $<_T=<$.  This isn't a very interesting tree, since it simply consists of a line of nodes.  However note that the height is $\omega$ even though no particular node has that height.

A more interesting tree using $\mathbb{Z}^+$ defines $m<_T n$ if $i^a=m$ and $i^b=n$ for some $i,a,b\in \mathbb{Z}^+\cup \{0\}$.  Then $1$ is the root, and all numbers which are not powers of another number are in $T_1$.  Then all squares (which are not also fourth powers) for $T_2$, and so on.

To illustrate the concept of a cofinal branch, observe that for any limit ordinal $\kappa$ we can construct a $\kappa$-tree which has no cofinal branches.  We let $T=\{(\alpha,\beta)|\alpha<\beta<\kappa\}$ and $(\alpha_1,\beta_1)<_T(\alpha_2,\beta_2)\leftrightarrow \alpha_1<\alpha_2 \wedge \beta_1=\beta_2$.  The tree then has $\kappa$ disjoint branches, each consisting of the set $\{(\alpha,\beta)|\alpha<\beta\}$ for some $\beta<\kappa$.  No branch is cofinal, since each branch is capped at $\beta$ elements, but for any $\gamma<\kappa$, there is a branch of height $\gamma+1$.  Hence the supremum of the heights is $\kappa$.
%%%%%
%%%%%
\end{document}
