\documentclass[12pt]{article}
\usepackage{pmmeta}
\pmcanonicalname{DeBruijnErdHosTheorem}
\pmcreated{2013-03-22 15:11:21}
\pmmodified{2013-03-22 15:11:21}
\pmowner{marijke}{8873}
\pmmodifier{marijke}{8873}
\pmtitle{De Bruijn--Erd\H{o}s theorem}
\pmrecord{7}{36945}
\pmprivacy{1}
\pmauthor{marijke}{8873}
\pmtype{Theorem}
\pmcomment{trigger rebuild}
\pmclassification{msc}{05B25}
%\pmkeywords{incidence}
%\pmkeywords{finite geometry}
\pmrelated{FiniteProjectivePlane4}
\pmrelated{IncidenceStructures}
\pmrelated{Geometry}
\pmrelated{LinearSpace2}

\usepackage{amssymb}
% \usepackage{amsmath}
% \usepackage{amsfonts}

% used for TeXing text within eps files
%\usepackage{psfrag}
% need this for including graphics (\includegraphics)
%\usepackage{graphicx}

% for neatly defining theorems and propositions
%\usepackage{amsthm}
% making logically defined graphics
%%%\usepackage{xypic}

% there are many more packages, add them here as you need them

% define commands here %%%%%%%%%%%%%%%%%%%%%%%%%%%%%%%%%%
% portions from
% makra.sty 1989-2005 by Marijke van Gans %
                                          %          ^ ^
\catcode`\@=11                            %          o o
                                          %         ->*<-
                                          %           ~
%%%% CHARS %%%%%%%%%%%%%%%%%%%%%%%%%%%%%%%%%%%%%%%%%%%%%%

                        %    code    char  frees  for

\let\Para\S             %    \Para     ÃÃÃÃÃÃÃÃÃÃÃÃÃÃÃÃÃÃÃÃÃÃÃÃÃÃÃÃÃÃÃÃÃÃÃÃÃÃÃÃÃÃÃÃÃÃÃÃÃÃÃÃÃÃÃÃÃÃÃÃÃÃÃÃç   \S \scriptstyle
\let\Pilcrow\P          %    \Pilcrow  ÃÃÃÃÃÃÃÃÃÃÃÃÃÃÃÃÃÃÃÃÃÃÃÃÃÃÃÃÃÃÃÃÃÃÃÃÃÃÃÃÃÃÃÃÃÃÃÃÃÃÃÃÃÃÃÃÃÃÃÃÃÃÃÃö   \P
\mathchardef\pilcrow="227B

\mathchardef\lt="313C   %    \lt       <   <     bra
\mathchardef\gt="313E   %    \gt       >   >     ket

\let\bs\backslash       %    \bs       \
\let\us\_               %    \us       _     \_  ...

%%%% DIACRITICS %%%%%%%%%%%%%%%%%%%%%%%%%%%%%%%%%%%%%%%%%

%let\udot\d             % under-dot (text mode), frees \d
\let\odot\.             % over-dot (text mode),  frees \.
%let\hacek\v            % hacek (text mode),     frees \v
%let\makron\=           % makron (text mode),    frees \=
%let\tilda\~            % tilde (text mode),     frees \~
\let\uml\"              % umlaut (text mode),    frees \"

%def\ij/{i{\kern-.07em}j}
%def\trema#1{\discretionary{-}{#1}{\uml #1}}

%%%% amssymb %%%%%%%%%%%%%%%%%%%%%%%%%%%%%%%%%%%%%%%%%%%%

\let\le\leqslant
\let\ge\geqslant
%let\prece\preceqslant
%let\succe\succeqslant

%%%% USEFUL MISC %%%%%%%%%%%%%%%%%%%%%%%%%%%%%%%%%%%%%%%%

%def\C++{C$^{_{++}}$}

%let\writelog\wlog
%def\wl@g/{{\sc wlog}}
%def\wlog{\@ifnextchar/{\wl@g}{\writelog}}

%def\org#1{\lower1.2pt\hbox{#1}} 
% chem struct formulae: \bs, --- /  \org{C} etc. 

%%%% USEFUL INTERNAL LaTeX STUFF %%%%%%%%%%%%%%%%%%%%%%%%

%let\Ifnextchar=\@ifnextchar
%let\Ifstar=\@ifstar
%def\currsize{\@currsize}

%%%% KERNING, SPACING, BREAKING %%%%%%%%%%%%%%%%%%%%%%%%%

\def\comma{,\,\allowbreak}

%def\qqquad{\hskip3em\relax}
%def\qqqquad{\hskip4em\relax}
%def\qqqqquad{\hskip5em\relax}
%def\qqqqqquad{\hskip6em\relax}
%def\qqqqqqquad{\hskip7em\relax}
%def\qqqqqqqquad{\hskip8em\relax}

%%%% LAYOUT %%%%%%%%%%%%%%%%%%%%%%%%%%%%%%%%%%%%%%%%%%%%%

%%%% COUNTERS %%%%%%%%%%%%%%%%%%%%%%%%%%%%%%%%%%%%%%%%%%%

%let\addtoreset\@addtoreset
%{A}{B} adds A to list of counters reset to 0
% when B is \refstepcounter'ed (see latex.tex)
%
%def\numbernext#1#2{\setcounter{#1}{#2}\addtocounter{#1}{\m@ne}}

%%%% EQUATIONS %%%%%%%%%%%%%%%%%%%%%%%%%%%%%%%%%%%%%%%%%%

%%%% LEMMATA %%%%%%%%%%%%%%%%%%%%%%%%%%%%%%%%%%%%%%%%%%%%

%%%% DISPLAY %%%%%%%%%%%%%%%%%%%%%%%%%%%%%%%%%%%%%%%%%%%%

%%%% MATH LAYOUT %%%%%%%%%%%%%%%%%%%%%%%%%%%%%%%%%%%%%%%%

\let\D\displaystyle
\let\T\textstyle
\let\S\scriptstyle
\let\SS\scriptscriptstyle

% array:
%def\<#1:{\begin{array}{#1}}
%def\>{\end{array}}

% array using [ ] with rounded corners:
%def\[#1:{\left\lgroup\begin{array}{#1}} 
%def\]{\end{array}\right\rgroup}

% array using ( ):
%def\(#1:{\left(\begin{array}{#1}}
%def\){\end{array}\right)}

%def\hh{\noalign{\vskip\doublerulesep}}

%%%% MATH SYMBOLS %%%%%%%%%%%%%%%%%%%%%%%%%%%%%%%%%%%%%%%

%def\d{\mathord{\rm d}}                      % d as in dx
%def\e{{\rm e}}                              % e as in e^x

%def\Ell{\hbox{\it\char`\$}}

\def\sfmath#1{{\mathchoice%
{{\sf #1}}{{\sf #1}}{{\S\sf #1}}{{\SS\sf #1}}}}
\def\Stalkset#1{\sfmath{I\kern-.12em#1}}
\def\Bset{\Stalkset B}
\def\Nset{\Stalkset N}
\def\Rset{\Stalkset R}
\def\Hset{\Stalkset H}
\def\Fset{\Stalkset F}
\def\kset{\Stalkset k}
\def\In@set{\raise.14ex\hbox{\i}\kern-.237em\raise.43ex\hbox{\i}}
\def\Roundset#1{\sfmath{\kern.14em\In@set\kern-.4em#1}}
\def\Qset{\Roundset Q}
\def\Cset{\Roundset C}
\def\Oset{\Roundset O}
\def\Zset{\sfmath{Z\kern-.44emZ}}

% \frac overwrites LaTeX's one (use TeX \over instead)
%def\fraq#1#2{{}^{#1}\!/\!{}_{\,#2}}
\def\frac#1#2{\mathord{\mathchoice%
{\T{#1\over#2}}
{\T{#1\over#2}}
{\S{#1\over#2}}
{\SS{#1\over#2}}}}
%def\half{\frac12}

\mathcode`\<="4268         % < now is \langle, \lt is <
\mathcode`\>="5269         % > now is \rangle, \gt is >

%def\biggg#1{{\hbox{$\left#1\vbox %to20.5\p@{}\right.\n@space$}}}
%def\Biggg#1{{\hbox{$\left#1\vbox %to23.5\p@{}\right.\n@space$}}}

\let\epsi=\varepsilon
\def\omikron{o}

\def\Alpha{{\rm A}}
\def\Beta{{\rm B}}
\def\Epsilon{{\rm E}}
\def\Zeta{{\rm Z}}
\def\Eta{{\rm H}}
\def\Iota{{\rm I}}
\def\Kappa{{\rm K}}
\def\Mu{{\rm M}}
\def\Nu{{\rm N}}
\def\Omikron{{\rm O}}
\def\Rho{{\rm P}}
\def\Tau{{\rm T}}
\def\Ypsilon{{\rm Y}} % differs from \Upsilon
\def\Chi{{\rm X}}

%def\dg{^{\circ}}                   % degrees

%def\1{^{-1}}                       % inverse

\def\*#1{{\bf #1}}                  % boldface e.g. vector
%def\vi{\mathord{\hbox{\bf\i}}}     % boldface vector \i
%def\vj{\mathord{\,\hbox{\bf\j}}}   % boldface vector \j

%def\union{\mathbin\cup}
%def\isect{\mathbin\cap}

\let\so\Longrightarrow
\let\oso\Longleftrightarrow
\let\os\Longleftarrow

% := and :<=>
%def\isdef{\mathrel{\smash{\stackrel{\SS\rm def}{=}}}}
%def\iffdef{\mathrel{\smash{stackrel{\SS\rm def}{\oso}}}}

\def\isdef{\mathrel{\mathop{=}\limits^{\smash{\hbox{\tiny def}}}}}
%def\iffdef{\mathrel{\mathop{\oso}\limits^{\smash{\hbox{\tiny %def}}}}}

%def\tr{\mathop{\rm tr}}            % tr[ace]
%def\ter#1{\mathop{^#1\rm ter}}     % k-ter[minant]

%let\.=\cdot
%let\x=\times                % ><ÃÃÃÃÃÃÃÃÃÃÃÃÃÃÃÃ (direct product)

%def\qed{ ${\S\circ}\!{}^\circ\!{\S\circ}$}
%def\qed{\vrule height 6pt width 6pt depth 0pt}

%def\edots{\mathinner{\mkern1mu
%   \raise7pt\vbox{\kern7pt\hbox{.}}\mkern1mu   %  .shorter
%   \raise4pt\hbox{.}\mkern1mu                  %     .
%   \raise1pt\hbox{.}\mkern1mu}}                %        .
%def\fdots{\mathinner{\mkern1mu
%   \raise7pt\vbox{\kern7pt\hbox{.}}            %   . ~45ÃÃÃÃÃÃÃÃÃÃÃÃÃÃÃÃð
%   \raise4pt\hbox{.}                           %     .
%   \raise1pt\hbox{.}\mkern1mu}}                %       .

\def\mod#1{\allowbreak \mkern 10mu({\rm mod}\,\,#1)}
% redefines TeX's one using less space

%def\int{\intop\displaylimits}
%def\oint{\ointop\displaylimits}

%def\intoi{\int_0^1}
%def\intall{\int_{-\infty}^\infty}

%def\su#1{\mathop{\sum\raise0.7pt\hbox{$\S\!\!\!\!\!#1\,$}}}

%let\frakR\Re
%let\frakI\Im
%def\Re{\mathop{\rm Re}\nolimits}
%def\Im{\mathop{\rm Im}\nolimits}
%def\conj#1{\overline{#1\vphantom1}}
%def\cj#1{\overline{#1\vphantom+}}

%def\forAll{\mathop\forall\limits}
%def\Exists{\mathop\exists\limits}

%%%% PICTURES %%%%%%%%%%%%%%%%%%%%%%%%%%%%%%%%%%%%%%%%%%%

\def\cent{\makebox(0,0)}

%def\node{\circle*4}
%def\nOde{\circle4}

%%%% REFERENCES %%%%%%%%%%%%%%%%%%%%%%%%%%%%%%%%%%%%%%%%%

%def\opcit{[{\it op.\,cit.}]}
\def\bitem#1{\bibitem[#1]{#1}}
\def\name#1{{\sc #1}}
\def\book#1{{\sl #1\/}}
\def\paper#1{``#1''}
\def\mag#1{{\it #1\/}}
\def\vol#1{{\bf #1}}
\def\isbn#1{{\small\tt ISBN\,\,#1}}
\def\seq#1{{\small\tt #1}}
%def\url<{\verb>}
%def\@cite#1#2{[{#1\if@tempswa\ #2\fi}]}

%%%% VERBATIM CODE %%%%%%%%%%%%%%%%%%%%%%%%%%%%%%%%%%%%%%

%def\"{\verb"}

%%%% AD HOC %%%%%%%%%%%%%%%%%%%%%%%%%%%%%%%%%%%%%%%%%%%%%

%%%% WORDS %%%%%%%%%%%%%%%%%%%%%%%%%%%%%%%%%%%%%%%%%%%%%%

% \hyphenation{pre-sent pre-sents pre-sent-ed pre-sent-ing
% re-pre-sent re-pre-sents re-pre-sent-ed re-pre-sent-ing
% re-fer-ence re-fer-ences re-fer-enced re-fer-encing
% ge-o-met-ry re-la-ti-vi-ty Gauss-ian Gauss-ians
% Des-ar-gues-ian}

%def\oord/{o{\trema o}rdin\-ate}
% usage: C\oord/s, c\oord/.
% output: co\"ord... except when linebreak at co-ord...

%%%%%%%%%%%%%%%%%%%%%%%%%%%%%%%%%%%%%%%%%%%%%%%%%%%%%%%%%

                                          %
                                          %          ^ ^
\catcode`\@=12                            %          ` '
                                          %         ->*<-
                                          %           ~
\begin{document}
\PMlinkescapeword{collection}
\PMlinkescapephrase{finite plane}
\PMlinkescapeword{incident}
\PMlinkescapeword{order}
\PMlinkescapeword{satisfy}
\PMlinkescapeword{simple}
\PMlinkescapeword{structure}
\PMlinkescapeword{line}
\PMlinkescapeword{circle}

\subsection*{De Bruijn--Erd\H{o}s theorem}

Let $S$ be a set of $n\ne0$ `points', say $\{L_1,L_2,\dots L_n\}$, and
let $\Lambda_1$, $\Lambda_2$, \dots\ $\Lambda_\nu$ be $\nu$ different subsets
of $S$ of which any two have exactly one point in common. Now the {\bf De Bruijn--Erd\H{o}s theorem} says that $\nu\le n$, and that if $\nu=n$ then (up to renumbering) at least one of the following three must be true:
%
\begin{itemize}

\item $\Lambda_i=\{L_i,L_n\}$ for $i\ne n$, and $\Lambda_n=\{L_n\}$

\item $\Lambda_i=\{L_i,L_n\}$ for $i\ne n$, and $\Lambda_n=\{L_1,L_2,\dots
L_{n-1}\}$

\item $n$ is of the form $q^2+q+1$, each $\Lambda_i$ contains $q+1$ points,
and each point is contained in $q+1$ of the $\Lambda_i$

\end{itemize}
%
and we recognise the last case as \PMlinkid{finite projective planes}{6943} of order $q$. For $n=1$ ($q=0$) the first and last case overlap, and for $n=3$ ($q=1$) the last two cases do.  The second case is also known as a near-pencil.  The last two cases together are examples of linear spaces.  

To exclude the first two cases one usually defines projective planes to satisfy some non-triviality conditions; unfortunately that also excludes projective planes of order 0~and~1.
%
\begin{center}
\begin{picture}(200,78)(-100,0)

\put(-122,+30){\begin{picture}(100,48)(0,0)
\put(  0,+32){\circle*{4}}
\put(-40,  0){\circle*{4}}
\put(-24,  0){\circle*{4}}
\put( -8,  0){\circle*{4}}
\put( +8,  0){\circle*{4}}
\put(+24,  0){\circle*{4}}
\put(+40,  0){\circle*{4}}
\put(-47, -5.6){\line(+5,+4){54}}
\put(-29.25,-7){\line(+3,+4){34.5}}
\put(-10, -8){\line(+1,+4){12}}
\put(+10, -8){\line(-1,+4){12}}
\put(+29.25,-7){\line(-3,+4){34.5}}
\put(+47,-5.6){\line(-5,+4){54}}
\put(-19,+32){\line(+1, 0){38}}
\end{picture}}

\put(+18,+30){\begin{picture}(100,48)(0,0)
\put(  0,+32){\circle*{4}}
\put(-40,  0){\circle*{4}}
\put(-24,  0){\circle*{4}}
\put( -8,  0){\circle*{4}}
\put( +8,  0){\circle*{4}}
\put(+24,  0){\circle*{4}}
\put(+40,  0){\circle*{4}}
\put(-47, -5.6){\line(+5,+4){54}}
\put(-29.25,-7){\line(+3,+4){34.5}}
\put(-10, -8){\line(+1,+4){12}}
\put(+10, -8){\line(-1,+4){12}}
\put(+29.25,-7){\line(-3,+4){34.5}}
\put(+47, -5.6){\line(-5,+4){54}}
\put(-49,  0){\line(+1, 0){98}}
\end{picture}}

\put(+148,+30){\begin{picture}(100,48)(0,0)
\put(  0,+12.6){\circle{25}}
\put(-24,  0){\circle*{4}}
\put(+24,  0){\circle*{4}}
\put(  0,+36){\circle*{4}}
\put(  0,+12){\circle*{4}}
\put(  0,  0){\circle*{4}}
\put(-12,+18){\circle*{4}}
\put(+12,+18){\circle*{4}}
\put(-31,  0){\line(+1, 0){62}}
\put(-28, -6){\line(+2,+3){32}}
\put(+28, -6){\line(-2,+3){32}}
\put(-30, -3){\line(+2,+1){48}}
\put(+30, -3){\line(-2,+1){48}}
\put(  0,+43){\line( 0,-1){50}}
\end{picture}}

\put(0,10){\cent[t]{\it The three cases of De\,Bruijn--Erd\H{o}s drawn for
$n=\nu=7$ ($q=2$)}}

\end{picture}
\end{center}
%
The formulation above was found in the literature \cite{Cam94}. Naturally, if
the $L_i$ are points we tend to interpret the subsets $\Lambda_\iota$
as lines. However, interpreted in this way the condition that every two
$\Lambda$ share an $L$ says rather more than is needed for the structure to
be a finite plane (\PMlinkid{linear space}{3509}) as it insists that no two lines are parallel. The absence of the dual condition, for two $L$ to share a $\Lambda$, actually means we have {\em too little\/} for the structure to be a finite plane (linear space), as for two points we may not have a line through them. And indeed, while the second and third cases are finite planes without parallel lines, the first case is not a plane.

\clearpage
\subsection*{Erd\H{o}s--De Bruijn theorem}

A dual formulation (which we could whimsically call the {\bf Erd\H{o}s--De
Bruijn Theorem}) remedies both under- and over-specification for a plane.
Indeed, the conditions are now exactly tailored to make the structure a finite
plane (with some parallel lines in the first of the three cases).
%
\begin{center}
\begin{picture}(200,78)(-100,-90)

\put(-122,-30){\begin{picture}(100,48)(0,0)
\put(-56,  0){\circle*{4}}
\put(-40,  0){\circle*{4}}
\put(-24,  0){\circle*{4}}
\put( -8,  0){\circle*{4}}
\put( +8,  0){\circle*{4}}
\put(+24,  0){\circle*{4}}
\put(+40,  0){\circle*{4}}
\put(-40, +8){\line( 0,-1){46}}
\put(-24, +8){\line( 0,-1){46}}
\put( -8, +8){\line( 0,-1){46}}
\put( +8, +8){\line( 0,-1){46}}
\put(+24, +8){\line( 0,-1){46}}
\put(+40, +8){\line( 0,-1){46}}
\put(-64,  0){\line(+1, 0){112}}
\end{picture}}

\put(+18,-30){\begin{picture}(100,48)(0,0)
\put(  0,-32){\circle*{4}}
\put(-40,  0){\circle*{4}}
\put(-24,  0){\circle*{4}}
\put( -8,  0){\circle*{4}}
\put( +8,  0){\circle*{4}}
\put(+24,  0){\circle*{4}}
\put(+40,  0){\circle*{4}}
\put(-47, +5.6){\line(+5,-4){54}}
\put(-29.25,+7){\line(+3,-4){34.5}}
\put(-10, +8){\line(+1,-4){12}}
\put(+10, +8){\line(-1,-4){12}}
\put(+29.25,+7){\line(-3,-4){34.5}}
\put(+47, +5.6){\line(-5,-4){54}}
\put(-49,  0){\line(+1, 0){98}}
\end{picture}}
\put(+148,-30){\begin{picture}(100,48)(0,0)
\put(  0,-12.6){\circle{25}}
\put(-24,  0){\circle*{4}}
\put(+24,  0){\circle*{4}}
\put(  0,-36){\circle*{4}}
\put(  0,-12){\circle*{4}}
\put(  0,  0){\circle*{4}}
\put(-12,-18){\circle*{4}}
\put(+12,-18){\circle*{4}}
\put(-31,  0){\line(+1, 0){62}}
\put(-28, +6){\line(+2,-3){32}}
\put(+28, +6){\line(-2,-3){32}}
\put(-30, +3){\line(+2,-1){48}}
\put(+30, +3){\line(-2,-1){48}}
\put(  0,-43){\line( 0,+1){50}}
\end{picture}}

\put(0,-80){\cent[t]{\it The three cases of Erd\H{o}s--De\,Bruijn drawn for
$\nu=n=7$ ($q=2$)}}

\end{picture}
\end{center}
%
Let $\Sigma$ be a set of $\nu\ne0$ `points', say $\{\Lambda_1,\Lambda_2,\dots
\Lambda_\nu\}$, and let $L_1$, $L_2$, \dots\ $L_n$ be $n$ subsets of $\Sigma$
(`lines') such that for any two points there is a unique $L_i$ that contains
them both. We must now be careful about the points (the former lines) being
``different'' (this was easier to formulate in the previous version, in which
form it was a {\em simple\/} incidence structure): this condition now takes the
form that no two points must have the same collection of lines that are
incident with them. Then $n\ge\nu$, and if $n=\nu$ then (up to renumbering)
at least one of the following three must be true:
%
\begin{itemize}

\item $L_i=\{\Lambda_i\}$ for $i\ne n$, and $L_n=\{\Lambda_1,\Lambda_2,\dots
\Lambda_n\}$

\item $L_i=\{\Lambda_i,\Lambda_n\}$ for $i\ne n$, and $L_n=\{\Lambda_1,\Lambda_2
\Lambda_{n-1}\}$

\item $n$ is of the form $q^2+q+1$, each $L_i$ contains $q+1$ points,
and each point is contained in $q+1$ of the $L_i$

\end{itemize}
%
For a proof of the theorem (in the former version) see e.g.\ Cameron \cite{Cam94}.

\clearpage
\raggedright
\begin{thebibliography}{MST73}

\bitem{Cam94} \name{Peter J.\,Cameron},
              \book{Combinatorics: topics, techniques, algorithms},\\
              Camb.~Univ.~Pr.~1994, \isbn{0\,521\,45761\,0}\\
{\tt\PMlinkexternal{http://www.maths.qmul.ac.uk/~pjc/comb/}{http://www.maths.qmul.ac.uk/~pjc/comb/}}
              (solutions, errata \&c.)

\end{thebibliography}
%%%%%
%%%%%
\end{document}
