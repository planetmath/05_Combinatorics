\documentclass[12pt]{article}
\usepackage{pmmeta}
\pmcanonicalname{SpernersTheorem}
\pmcreated{2013-03-22 13:51:57}
\pmmodified{2013-03-22 13:51:57}
\pmowner{bbukh}{348}
\pmmodifier{bbukh}{348}
\pmtitle{Sperner's theorem}
\pmrecord{7}{34606}
\pmprivacy{1}
\pmauthor{bbukh}{348}
\pmtype{Theorem}
\pmcomment{trigger rebuild}
\pmclassification{msc}{05D05}
\pmclassification{msc}{06A07}
%\pmkeywords{antichain}
%\pmkeywords{maximal antichain}
\pmrelated{LYMInequality}
\pmdefines{Sperner family}

\endmetadata

\usepackage{amssymb}
\usepackage{amsmath}
\usepackage{amsfonts}
\usepackage{amsthm}
\newtheorem*{theorem}{Theorem}
\newcommand*{\abs}[1]{\left\lvert #1\right\rvert}
\newcommand*{\floor}[1]{\left\lfloor #1\right\rfloor}

\makeatletter
\@ifundefined{bibname}{}{\renewcommand{\bibname}{References}}
\makeatother
\begin{document}
\PMlinkescapeword{refinements}
What is the size of the largest family $\mathcal{F}$ of subsets of
an $n$-element set such that no $A\in \mathcal{F}$ is a subset of
$B\in \cal{F}$? Sperner \cite{cite:sperner_antichain} gave an
answer in the following elegant theorem:
\begin{theorem}
For every family $\mathcal{F}$ of incomparable subsets of an
$n$-set, $\abs{\mathcal{F}}\leq \binom{n}{\floor{n/2}}$.
\end{theorem}
A family satisfying the conditions of Sperner's theorem is usually
called \emph{Sperner family} or \emph{antichain}. The later
terminology stems from the fact that subsets of a finite set
ordered by inclusion form a Boolean lattice.

There are many generalizations of Sperner's theorem. On one hand,
there are refinements like LYM inequality that strengthen the
theorem in various ways. On the other hand, there are
generalizations to posets other than the Boolean lattice. For a
comprehensive exposition of the topic one should consult a
well-written monograph by Engel\cite{cite:engel_sperner}.

\begin{thebibliography}{1}

\bibitem{cite:bollobas_comb}
B{\'e}la Bollob{\'a}s.
\newblock Combinatorics: Set systems, hypergraphs, families of vectors, and
  combinatorial probability.
\newblock 1986.
\newblock \PMlinkexternal{Zbl
  0595.05001}{http://www.emis.de/cgi-bin/zmen/ZMATH/en/quick.html?type=html&an%
=0595.05001}.

\bibitem{cite:engel_sperner}
Konrad Engel.
\newblock {\em Sperner theory}, volume~65 of {\em Encyclopedia of Mathematics
  and Its Applications}.
\newblock Cambridge University Press.
\newblock \PMlinkexternal{Zbl
  0868.05001}{http://www.emis.de/cgi-bin/zmen/ZMATH/en/quick.html?type=html&an%
=0868.05001}.

\bibitem{cite:sperner_antichain}
Emanuel Sperner.
\newblock {E}in {S}atz {\"u}ber {U}ntermengen einer endlichen {M}enge.
\newblock {\em Math. Z.}, 27:544--548, 1928.
\newblock \PMlinkexternal{Available online}{http://www.emis.de/cgi-bin/jfmen/MATH/JFM/quick.html?first=1&maxdocs=20&type=html&an=JFM\%2054.0090.06&format=complete} at
  \PMlinkexternal{JFM}{http://www.emis.de/projects/JFM/}.

\end{thebibliography}
%%%%%
%%%%%
\end{document}
