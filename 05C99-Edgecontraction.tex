\documentclass[12pt]{article}
\usepackage{pmmeta}
\pmcanonicalname{Edgecontraction}
\pmcreated{2013-03-22 12:31:43}
\pmmodified{2013-03-22 12:31:43}
\pmowner{rspuzio}{6075}
\pmmodifier{rspuzio}{6075}
\pmtitle{edge-contraction}
\pmrecord{5}{32769}
\pmprivacy{1}
\pmauthor{rspuzio}{6075}
\pmtype{Definition}
\pmcomment{trigger rebuild}
\pmclassification{msc}{05C99}
\pmrelated{TheoremOn3ConnectedGraphs}
\pmdefines{contraction}

% this is the default PlanetMath preamble.  as your knowledge
% of TeX increases, you will probably want to edit this, but
% it should be fine as is for beginners.

% almost certainly you want these
\usepackage{amssymb}
\usepackage{amsmath}
\usepackage{amsfonts}

% used for TeXing text within eps files
%\usepackage{psfrag}
% need this for including graphics (\includegraphics)
%\usepackage{graphicx}
% for neatly defining theorems and propositions
%\usepackage{amsthm}
% making logically defined graphics
%%%\usepackage{xypic} 

% there are many more packages, add them here as you need them

% define commands here
\begin{document}
Given an edge $xy$ of a graph $G$, the graph $G/xy$ is obtained from $G$ by \emph{contracting} the edge $xy$; that is, to get $G/xy$ we identify the vertices $x$ and $y$ and remove all loops and duplicate edges. A graph $G'$ obtained by a sequence of edge-contractions is said to be a \emph{contraction} of $G$.


\footnotesize{Adapted with permission of the author from \emph{\PMlinkescapetext{Modern Graph Theory}} by B\'{e}la Bollob\'{a}s, published by Springer-Verlag New York, Inc., 1998.}
%%%%%
%%%%%
\end{document}
