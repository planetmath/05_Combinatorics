\documentclass[12pt]{article}
\usepackage{pmmeta}
\pmcanonicalname{Cut}
\pmcreated{2013-03-22 13:00:53}
\pmmodified{2013-03-22 13:00:53}
\pmowner{vampyr}{22}
\pmmodifier{vampyr}{22}
\pmtitle{cut}
\pmrecord{5}{33398}
\pmprivacy{1}
\pmauthor{vampyr}{22}
\pmtype{Definition}
\pmcomment{trigger rebuild}
\pmclassification{msc}{05C20}
\pmrelated{MaximumFlowminimumCutTheorem}
\pmdefines{minimum cut}

\endmetadata

% this is the default PlanetMath preamble.  as your knowledge
% of TeX increases, you will probably want to edit this, but
% it should be fine as is for beginners.

% almost certainly you want these
\usepackage{amssymb}
\usepackage{amsmath}
\usepackage{amsfonts}

% used for TeXing text within eps files
%\usepackage{psfrag}
% need this for including graphics (\includegraphics)
%\usepackage{graphicx}
% for neatly defining theorems and propositions
%\usepackage{amsthm}
% making logically defined graphics
%%%\usepackage{xypic} 

% there are many more packages, add them here as you need them

% define commands here
\begin{document}
On a digraph, define a \emph{sink} to be a vertex with out-degree zero and a \emph{source} to be a vertex with in-degree zero.  Let $G$ be a digraph with non-negative weights and with exactly one sink and exactly one source.  A \emph{cut} $C$ on $G$ is a subset of the edges such that every path from the source to the sink passes through an edge in $C$.  In other words, if we remove every edge in $C$ from the graph, there is no longer a path from the source to the sink.

Define the weight of $C$ as
$$W_C = \sum_{e \in C} W(e)$$
where $W(e)$ is the weight of the edge $e$.

Observe that we may achieve a trivial cut by removing all the edges of $G$.  Typically, we are more interested in \emph{minimal cuts}, where the weight of the cut is minimized for a particular graph.
%%%%%
%%%%%
\end{document}
