\documentclass[12pt]{article}
\usepackage{pmmeta}
\pmcanonicalname{LeviCivitaPermutationSymbol}
\pmcreated{2013-03-22 13:31:29}
\pmmodified{2013-03-22 13:31:29}
\pmowner{matte}{1858}
\pmmodifier{matte}{1858}
\pmtitle{Levi-Civita permutation symbol}
\pmrecord{13}{34116}
\pmprivacy{1}
\pmauthor{matte}{1858}
\pmtype{Definition}
\pmcomment{trigger rebuild}
\pmclassification{msc}{05A10}
\pmrelated{KroneckerDelta}
\pmrelated{GeneralizedKroneckerDeltaSymbol}

% this is the default PlanetMath preamble.  as your knowledge
% of TeX increases, you will probably want to edit this, but
% it should be fine as is for beginners.

% almost certainly you want these
\usepackage{amssymb}
\usepackage{amsmath}
\usepackage{amsfonts}
\usepackage{amsthm}

\usepackage{mathrsfs}

% used for TeXing text within eps files
%\usepackage{psfrag}
% need this for including graphics (\includegraphics)
%\usepackage{graphicx}
% for neatly defining theorems and propositions
%
% making logically defined graphics
%%%\usepackage{xypic}

% there are many more packages, add them here as you need them

% define commands here

\newcommand{\sR}[0]{\mathbbmss{R}}
\newcommand{\sC}[0]{\mathbbmss{C}}
\newcommand{\sN}[0]{\mathbbmss{N}}
\newcommand{\sZ}[0]{\mathbbmss{Z}}

 \usepackage{bbm}
 \newcommand{\Z}{\mathbbmss{Z}}
 \newcommand{\C}{\mathbbmss{C}}
 \newcommand{\R}{\mathbbmss{R}}
 \newcommand{\Q}{\mathbbmss{Q}}



\newcommand*{\norm}[1]{\lVert #1 \rVert}
\newcommand*{\abs}[1]{| #1 |}



\newtheorem{thm}{Theorem}
\newtheorem{defn}{Definition}
\newtheorem{prop}{Proposition}
\newtheorem{lemma}{Lemma}
\newtheorem{cor}{Corollary}
\begin{document}
\begin{defn}
Let $k_i \in \{1,\cdots, n\}$ for all $i=1,\cdots ,n$.
The \emph{Levi-Civita permutation symbols} $\varepsilon_{k_1\cdots k_n}$ and $\varepsilon^{k_1\cdots k_n}$ are
defined as
$$\varepsilon_{k_1\cdots k_m}=\varepsilon^{k_1\cdots k_m}= \left\{  \begin {array}{ll} +1 & \mbox{when} \, \{l\mapsto k_l\} \mbox{ is an even permutation (of $\{1,\cdots, n\}$),} \\
-1 & \mbox{when} \, \{l\mapsto k_l\} \mbox{ is an odd permutation,} \\
   0 &  \mbox{otherwise, i.e., when\,} k_i = k_j, \ \mbox {for some } \ i\neq j.  \\
    \end{array} \right. $$
\end{defn}

The Levi-Civita permutation symbol is a special case of  the generalized
Kronecker delta symbol. Using this fact one can write the Levi-Civita permutation
symbol as the determinant of an $n\times n$ matrix consisting of traditional
delta symbols. See the entry on the generalized Kronecker symbol for details.
 

When using the Levi-Civita permutation symbol and the generalized Kronecker delta
symbol, the Einstein summation convention is usually employed. In the below, 
we shall also use this convention.

\subsubsection*{Properties}
\begin{itemize}
\item When $n=2$, we have for all $i,j,m,n$ in $\{1,2\}$, 
\begin{eqnarray}
\label{eq0}
\varepsilon_{ij} \varepsilon^{mn} &=& \delta_i^m \delta_j^n - \delta_i^n \delta_j^m, \\
\label{eq1}
\varepsilon_{ij} \varepsilon^{in} &=& \delta_j^n,\\
\label{eq2}
\varepsilon_{ij} \varepsilon^{ij} &=& 2.
\end{eqnarray}

\item When $n=3$, we have for all $i,j,k,m,n$ in $\{1,2,3\}$, 
\begin{eqnarray}
\label{eq3}
\varepsilon_{jmn} \varepsilon^{imn} &=& 2\delta^i_j, \\
\label{eq4}
\varepsilon_{ijk} \varepsilon^{ijk} &=& 6.
\end{eqnarray}
\end{itemize}

Let us prove these properties. The proofs are instructional since they
demonstrate typical argumentation methods for manipulating the
permutation symbols.

\emph{Proof.} For equation \ref{eq0}, let us first note that both sides
are antisymmetric with respect of $ij$ and $mn$. We therefore only need
to consider the case $i\neq j$ and $m\neq n$. By substitution, we see that
the equation holds for $\varepsilon_{12} \varepsilon^{12}$, i.e., for 
$i=m=1$ and $j=n=2$. (Both sides are then one). Since the equation is 
anti-symmetric in $ij$ and $mn$, any set of values for these can be 
reduced the above case (which holds). The equation 
thus holds for all values of $ij$ and $mn$.
Using equation \ref{eq0}, we have for equation \ref{eq1} 
\begin{eqnarray*}
\varepsilon_{ij}\varepsilon^{in} &=& \delta_i^i \delta_j^n - \delta^n_i \delta^i_j\\
&=& 2 \delta_j^n - \delta^n_j\\
&=& \delta_j^n. 
\end{eqnarray*}
Here we used the Einstein summation convention with $i$ going from $1$ to $2$. 
Equation \ref{eq2} follows similarly from equation \ref{eq1}.
To establish equation \ref{eq3}, let us first observe that both sides 
vanish when $i\neq j$. Indeed, if $i\neq j$, then one can not choose 
$m$ and $n$ such that both permutation symbols on the left are nonzero. Then, 
with $i=j$ fixed, there are only two ways to choose $m$ and $n$ from the remaining
two indices. For any such indices, we have
$\varepsilon_{jmn} \varepsilon^{imn} = (\varepsilon^{imn})^2 = 1$ (no summation),
and the result follows. The last property follows since $3!=6$ and for any 
distinct indices $i,j,k$ in $\{1,2,3\}$, we have 
$\varepsilon_{ijk} \varepsilon^{ijk}=1$ (no summation). $\Box$

\subsubsection*{Examples and Applications.} 
\begin{itemize}
\item The determinant of an $n\times n$ matrix $A=(a_{ij})$ can be written
as
$$ \det A = \varepsilon_{i_1\cdots i_n} a_{1i_1} \cdots a_{ni_n},$$
where each $i_l$ should be summed over $1,\ldots, n$. 
\item If $A=(A^1, A^2, A^3)$ and $B=(B^1, B^2, B^3)$ are vectors in 
$\sR^3$ (represented in some right hand oriented orthonormal basis), then 
the $i$th component of their cross product equals
$$ (A\times B)^i = \varepsilon^{ijk} A^j B^k.$$
For instance, the first component of $A\times B$ is
$A^2 B^3-A^3 B^2$. From the above expression for the cross product, 
it is clear that $A\times B = -B\times A$. 
Further, if $C=(C^1, C^2, C^3)$ is a vector like $A$ and $B$, then 
the triple scalar product equals
$$ A\cdot(B\times C) = \varepsilon^{ijk} A^i B^j C^k.$$
From this expression, it can be seen that the triple scalar product is 
antisymmetric when exchanging any adjacent arguments.
For example, $A\cdot(B\times C)= -B\cdot(A\times C)$. 
\item Suppose $F=(F^1, F^2, F^3)$ is a vector field defined on some 
open set of $\R^3$ with Cartesian coordinates $x=(x^1, x^2, x^3)$. Then 
the $i$th component of the curl of $F$ equals
$$ 
  (\nabla \times F)^i(x) = \varepsilon^{ijk}\frac{\partial}{\partial x^j} F^k(x).
$$
\end{itemize}
%%%%%
%%%%%
\end{document}
