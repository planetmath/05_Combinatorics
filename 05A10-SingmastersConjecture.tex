\documentclass[12pt]{article}
\usepackage{pmmeta}
\pmcanonicalname{SingmastersConjecture}
\pmcreated{2013-03-22 18:51:10}
\pmmodified{2013-03-22 18:51:10}
\pmowner{PrimeFan}{13766}
\pmmodifier{PrimeFan}{13766}
\pmtitle{Singmaster's conjecture}
\pmrecord{4}{41662}
\pmprivacy{1}
\pmauthor{PrimeFan}{13766}
\pmtype{Conjecture}
\pmcomment{trigger rebuild}
\pmclassification{msc}{05A10}

\endmetadata

% this is the default PlanetMath preamble.  as your knowledge
% of TeX increases, you will probably want to edit this, but
% it should be fine as is for beginners.

% almost certainly you want these
\usepackage{amssymb}
\usepackage{amsmath}
\usepackage{amsfonts}

% used for TeXing text within eps files
%\usepackage{psfrag}
% need this for including graphics (\includegraphics)
%\usepackage{graphicx}
% for neatly defining theorems and propositions
%\usepackage{amsthm}
% making logically defined graphics
%%%\usepackage{xypic}

% there are many more packages, add them here as you need them

% define commands here

\begin{document}
Conjecture. (David Singmaster). With the exception of the number 1, no positive integer appears in Pascal's triangle more than twelve times.

Numbering the top row of Pascal's triangle (the tip with the single instance of 1) as row 0, and leftmost column as row 0, it is clear that each integer $n > 2$ occurs at least twice, specifically, at positions $(n, 1)$ and $(n, n - 1)$. Singmaster was able to figure out that when there is a solution to $$n = { F_{2k} F_{2k + 1} \choose F_{2k - 1} F_{2k} - 1} - 1 = { F_{2k} F_{2k + 1} - 1 \choose F_{2k - 1} F_{2k}}$$ (with $F_i$ being the $i$th Fibonacci number) the number $n$ occurs six times in Pascal's triangle, and he showed that there are infinitely many such numbers. Empirical evidence suggests the actual maximum of instances for a number to occur in Pascal's triangle may be less than twelve: Pascal's triangle has been computed to millions of rows and no number (besides 1) has been encountered more than eight times.
%%%%%
%%%%%
\end{document}
