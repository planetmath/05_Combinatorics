\documentclass[12pt]{article}
\usepackage{pmmeta}
\pmcanonicalname{GraphHomeomorphism}
\pmcreated{2013-03-22 18:02:02}
\pmmodified{2013-03-22 18:02:02}
\pmowner{Ziosilvio}{18733}
\pmmodifier{Ziosilvio}{18733}
\pmtitle{graph homeomorphism}
\pmrecord{8}{40553}
\pmprivacy{1}
\pmauthor{Ziosilvio}{18733}
\pmtype{Definition}
\pmcomment{trigger rebuild}
\pmclassification{msc}{05C99}
\pmdefines{simple subdivision}

\endmetadata

% this is the default PlanetMath preamble.  as your knowledge
% of TeX increases, you will probably want to edit this, but
% it should be fine as is for beginners.

% almost certainly you want these
\usepackage{amssymb}
\usepackage{amsmath}
\usepackage{amsfonts}

% used for TeXing text within eps files
%\usepackage{psfrag}
% need this for including graphics (\includegraphics)
%\usepackage{graphicx}
% for neatly defining theorems and propositions
%\usepackage{amsthm}
% making logically defined graphics
%%%\usepackage{xypic}

% there are many more packages, add them here as you need them

% define commands here
\newcommand{\ie}{\textit{i.e.}}

\begin{document}
Let $G=(V,E)$ be a simple undirected graph.
A \emph{simple subdivision} is the replacement of an edge $(x,y)\in E$
with a pair of edges $(x,z),(z,y)$,
$z$ being a new vertex, \ie, $z\not\in V$.
The reverse operation of a simple subdivision
is an edge-contraction through a vertex of degree 2.

Two graphs $G_1$, $G_2$ are \emph{homeomorphic}
if $G_1$ can be transformed into $G_2$
via a finite sequence of simple subdivisions
and edge-contractions through vertices of degree 2.
It is easy to see that graph homeomorphism is an equivalence relation.

Equivalently, $G_1$ and $G_2$ are homeomorphic
if there exists a third graph $G_3$
such that both $G_1$ and $G_2$ can be obtained from $G_3$
via a finite sequence of edge-contractions through vertices of degree 2.

If a graph $G$ has a subgraph $H$
which is homeomorphic to a graph $G'$ having no vertices of degree 2,
then $G'$ is a minor of $G$.
The vice versa is not true:
as a counterexample, the Petersen graph
has $K_{5}$ as a minor, but no subgraph homeomorphic to $K_{5}$.
This happens because a graph homeomorphism
cannot change the number of vertices of degree $d\neq 2$:
since all the vertices of $K_{5}$ have degree 4
and all the vertices of the Petersen graph have degree 3,
no subgraph of the Petersen graph can be homeomorphic to $K_{5}$.

%%%%%
%%%%%
\end{document}
