\documentclass[12pt]{article}
\usepackage{pmmeta}
\pmcanonicalname{LocallyFiniteGraph}
\pmcreated{2013-03-22 16:00:51}
\pmmodified{2013-03-22 16:00:51}
\pmowner{yark}{2760}
\pmmodifier{yark}{2760}
\pmtitle{locally finite graph}
\pmrecord{19}{38049}
\pmprivacy{1}
\pmauthor{yark}{2760}
\pmtype{Definition}
\pmcomment{trigger rebuild}
\pmclassification{msc}{05C99}
\pmrelated{UniformlyLocallyFiniteGraph}

\endmetadata

\usepackage{amssymb}
\usepackage{amsmath}
\usepackage{amsfonts}

\usepackage[dvips,all]{xypic}

\begin{document}
\PMlinkescapeword{finite}
\PMlinkescapeword{infinite}
\PMlinkescapeword{length}
\PMlinkescapeword{lines}
\PMlinkescapeword{point}
\PMlinkescapeword{similar}
\PMlinkescapeword{unit}
\PMlinkescapephrase{locally finite}

A \emph{locally finite graph} is a graph in which every vertex has \PMlinkname{finite}{Finite} degree.
Note that any finite graph is locally finite;
however, infinite graphs can also be locally finite.
For example, consider the graph given by $\mathbb{Z}^{2}$,
where the points are the vertices and the line segments of unit length that connect vertices are edges:
\[
\xymatrix{
& \vdots \ar@{-}[d] & \vdots \ar@{-}[d] &  \\
\dots \ar@{-}[r] & \cdot \ar@{-}[r] \ar@{-}[d] & \cdot \ar@{-}[d] \ar@{-}[r] & \dots \\
\dots \ar@{-}[r] & \cdot \ar@{-}[r] \ar@{-}[d] & \cdot \ar@{-}[r] \ar@{-}[d] & \dots \\ & \vdots & \vdots }
\]

Note that every vertex has degree $4$ but that this is an infinite graph.

The $k$-\PMlinkname{regular}{RegularGraph} trees
form another class of examples of locally finite graphs.
A picture of a $2$-regular tree is given below

\[
\xymatrix{
& & & \cdot \\
& & \cdot \ar@{--}[ur] &  \\
& \cdot \ar@{-}[ur] & & \\
\cdot \ar@{-}[ur] \ar@{-}[dr] & & & \\
& \cdot \ar@{-}[dr] & & \\
& & \cdot \ar@{--}[dr] & \\
& & & \cdot }
\]

The solid unit length dashed are edges while the dots are vertices and the dashed lines are meant to denote continuation.
For a similar reason this is a locally finite graph.
%%%%%
%%%%%
\end{document}
