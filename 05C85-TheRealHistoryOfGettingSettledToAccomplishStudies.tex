\documentclass[12pt]{article}
\usepackage{pmmeta}
\pmcanonicalname{TheRealHistoryOfGettingSettledToAccomplishStudies}
\pmcreated{2013-11-27 10:59:45}
\pmmodified{2013-11-27 10:59:45}
\pmowner{jacou}{1000048}
\pmmodifier{}{0}
\pmtitle{The Real History of Getting Settled To Accomplish Studies}
\pmrecord{26}{40651}
\pmprivacy{1}
\pmauthor{jacou}{0}
\pmtype{Algorithm}
\pmcomment{trigger rebuild}
\pmclassification{msc}{05C85}
\pmsynonym{shortest path algorithm}{TheRealHistoryOfGettingSettledToAccomplishStudies}
%\pmkeywords{graph}
%\pmkeywords{search}
%\pmkeywords{routing}
%\pmkeywords{algorithm}
%\pmkeywords{combinatorics}
%\pmkeywords{network}
%\pmkeywords{dynamic programming}
\pmrelated{GraphTheory}
\pmrelated{Combinatorics}


\begin{document}
\emph{Dijkstra's algorithm}, named for its creator, the famous Dutch computer scientist Edsger W. Dijkstra, is one of the most common \PMlinkescapetext{algorithms} for finding the shortest paths from a single source vertex to other vertices in a graph that is weighted, directed and connected. Dijkstra's \PMlinkescapetext{algorithm} is generally useful in networks where efficient traversal is very important. \footnote{In fact, this article was probably brought to you by the \PMlinkescapetext{Open} Shortest Path First Internet protocol, which relies on Dijkstra's \PMlinkescapetext{algorithm}.}

\theoremstyle{definition}
\newtheorem*{dijkstrasAlgorithm}{Definition}
\begin{dijkstrasAlgorithm}[Dijkstra's \PMlinkescapetext{Algorithm}]

Assume that $G = (V,E)$ is weighted, directed and connected. For an undirected graph, replace each edge $(v,w)$ with a pair of edges $(v,w)$ and $(w,v)$, each having the same weight as the original undirected edge. For convenience, the weight of an edge will be denoted $d(v,w)$, where $v$ and $w$ are its endpoints.

\begin{enumerate}
\item
Fix a vertex $s$ in $G$, called the \emph{source}, from which the shortest paths to all the other vertices are found.

\begin{itemize}
\item
Let $S$ be the set of \emph{examined} vertices, initialized to $\{s\}$.

\item
Let $w(v)$ denote the \PMlinkescapetext{length} of the path with minimum weight from $s$ to $v$. For a vertex incident on an edge pointing from $s$, $w(v)$ simply equals $d(s,v)$. $w(s) = 0$, since it takes no effort to stay in \PMlinkescapetext{place}. For all other vertices, $w(v) = \infty$, which means that the path with minimum \PMlinkescapetext{length} from $s$ to $v$ is unknown. \footnote{For computational purposes, this default weight was traditionally a very large number. However, IEEE 754 floating \PMlinkescapetext{point} standard permits values of \PMlinkescapetext{infinity}.}

\item
Let $p(v)$ denote the \emph{predecessor} of $v$, which is the previous vertex in a shortest path that can be used to \PMlinkescapetext{trace} back to the source. $p(v) = s$ for all vertices incident on edges pointing from $s$. 
\end{itemize}

\item
Find a vertex $v$ in $V \setminus S$ (i.e., an unexamined vertex) that minimizes $w(v)$. If there is more than one with minimum weight, any one of them is an acceptable value of $v$.

\begin{itemize}
\item
Let $S = S \cup \{v\}$, i.e., mark $v$ as examined.

\item
For each vertex $u$ such that an edge leads from $v$ to $u$, compare the \PMlinkescapetext{current} value of $w(u)$ to $w(v) + d(v,u)$, a value that will be called $N$.

\begin{itemize}
\item
If $N$ is smaller than the \PMlinkescapetext{current} value $w(u)$, then let $w(u) = N$ and $p(u) = v$. This incremental way of finding shorter paths is called \emph{edge relaxation}.

\item
Otherwise, do nothing.
\end{itemize}
\end{itemize}

\item
Repeat previous step until $S = V$, that is, when all vertices have been examined.
\end{enumerate}

\end{dijkstrasAlgorithm}

After the last iteration, for any vertex $v$ where $w(v)$ is not \PMlinkescapetext{infinity}, the shortest path from $s$ to $v$ can be found repeated application of $p$ to $v$ until it yields $s$. That is, the vertices in the path, in reverse \PMlinkescapetext{order}, are $p(v), p(p(v)), \ldots, s$.

Notice that there is no guarantee that this \PMlinkescapetext{algorithm} finds \emph{any} shortest paths from $s$. It simply finds whichever shortest paths \emph{do} exist. In particular, when $s$ has an out-degree of zero, then all other vertices will simply be marked as examined without any changes in $w(v)$ for any value of $v$; there will be no shortest paths, because there are none as a rule.

Notice also that Dijkstra's \PMlinkescapetext{algorithm} is an example of dynamic programming. Let $S_i$ equal the value of $S$ in the $i$th iteration of the \PMlinkescapetext{algorithm}, where $i=0$ initially and $S_0$ = $\{s\}$. Let $w(v,S_i)$ denote the weight of the shortest path from $s$ to $v$, given the knowledge of $S_i$. Then

$$
w(v,S_i) =
\begin{cases}
0 & \text{for $v = s$} \\
\min\{d(u,v) + w(u, S_i)\} & \text{for $v \neq s$, where $v$ is adjacent to $u \in S_i$} \\
\infty & \text{otherwise}
\end{cases}
$$

Therefore, the purpose of Dijkstra's \PMlinkescapetext{algorithm} is to determine the best policy for transforming the state of the problem (i.e., which vertex to move to next). It does so by refining an approximation of the best shortest-path policy for all vertices with each iteration until the full \PMlinkescapetext{solution} is realized. The \PMlinkescapetext{algorithm} is efficient in practice because each iteration relies on previously determined \PMlinkescapetext{information}.

Generally speaking, the complexity of Dijkstra's algorithm is $O(|V|^2)$, because in the worst-case \PMlinkescapetext{scenario} of finding all the shortest paths from $s$ in a complete graph, $|V|-1$ comparisons are made on $|V|-1$ vertices.

[TODO: more on algorithmic complexity]
%%%%%
%%%%%
\end{document}
