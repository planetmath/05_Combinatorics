\documentclass[12pt]{article}
\usepackage{pmmeta}
\pmcanonicalname{BipartiteMatching}
\pmcreated{2013-03-22 12:40:08}
\pmmodified{2013-03-22 12:40:08}
\pmowner{mathcam}{2727}
\pmmodifier{mathcam}{2727}
\pmtitle{bipartite matching}
\pmrecord{7}{32942}
\pmprivacy{1}
\pmauthor{mathcam}{2727}
\pmtype{Definition}
\pmcomment{trigger rebuild}
\pmclassification{msc}{05C70}
\pmrelated{Saturate}
\pmdefines{complete matching}

\endmetadata

% this is the default PlanetMath preamble.  as your knowledge
% of TeX increases, you will probably want to edit this, but
% it should be fine as is for beginners.

% almost certainly you want these
\usepackage{amssymb}
\usepackage{amsmath}
\usepackage{amsfonts}

% used for TeXing text within eps files
%\usepackage{psfrag}
% need this for including graphics (\includegraphics)
\usepackage{graphicx}
% for neatly defining theorems and propositions
%\usepackage{amsthm}
% making logically defined graphics
%%%\usepackage{xypic} 

% there are many more packages, add them here as you need them

% define commands here
\begin{document}
A matching on a bipartite graph is called a \emph{bipartite matching}.  Bipartite matchings have many interesting properties.

\paragraph{Matrix form}

Suppose we have a bipartite graph $G$ and we partition the vertices into two sets, $V_1$ and $V_2$, of the same colour.  We may then represent the graph with a simplified adjacency matrix with $|V_1|$ rows and $|V_2|$ columns containing a $1$ where an edge joins the corresponding vertices and a $0$ where there is no edge.

We say that two $1$s in the matrix are \emph{line-independent} if they are not in the same row or column.  Then a matching on the graph with be a subset of the $1$s in the matrix that are all line-independent.

For example, consider this bipartite graph (the thickened edges are a matching):

\begin{center}
\includegraphics[scale=1.0]{bipartite3.eps}
\end{center}

The graph could be represented as the matrix
$$\begin{pmatrix}
1* & 1 & 0 & 1 & 0 \\
1 & 0 & 1* & 0 & 0 \\
0 & 0 & 1 & 0 & 1* \\
0 & 1* & 1 & 0 & 1 \\
\end{pmatrix}$$
where a $1*$ indicates an edge in the matching.  Note that all the starred $1$s are line-independent.

A \emph{complete matching} on a bipartite graph $G(V_1, V_2, E)$ is one that saturates all of the vertices in $V_1$.

\paragraph{Systems of distinct representatives}

A system of distinct representatives is equivalent to a maximal matching on some bipartite graph.  Let $V_1$ and $V_2$ be the two sets of vertices in the graph with $|V_1| \leq |V_2|$.  Consider the set $\{ v \in V_1 : \Gamma(v) \}$, which includes the neighborhood of every vertex in $V_1$.  An SDR for this set will be a unique choice of an vertex in $V_2$ for each vertex in $V_1$.  There must be an edge joining these vertices; the set of all such edges forms a matching.

Consider the sets
\begin{eqnarray*}
S_1 &=& \{A, B, D\} \\
S_2 &=& \{A,C \} \\
S_3 &=& \{C,E \} \\
S_4 &=& \{B,C,E \} \\
\end{eqnarray*}

One SDR for these sets is
\begin{eqnarray*}
A &\in& S_1 \\
C &\in& S_2 \\
E &\in& S_3 \\
B &\in& S_4 \\
\end{eqnarray*}

Note that this is the same matching on the graph shown above.

\paragraph{Finding Bipartite Matchings}

One method for finding maximal bipartite matchings involves using a network flow algorithm.  Before using it, however, we must modify the graph.

Start with a bipartite graph $G$.  As usual, we consider the two sets of vertices $V_1$ and $V_2$.  Replace every edge in the graph with a directed arc from $V_1$ to $V_2$ of capacity $L$, where $L$ is some large integer.

Invent two new vertices: the source and the sink.  Add a directed arc of capacity $1$ from the source to each vertex in $V_1$.  Likewise, add a directed arc of capacity $1$ from each vertex in $V_2$ to the sink.

Now find the maximum flow from the source to the sink.  The total weight of this flow will be the \PMlinkescapeword{size} of the maximum matching on $G$.  Similarly, the set of edges with non-zero flow will constitute a matching.

There also exist \PMlinkname{algorithms specifically for finding bipartite matchings}{MaximalBipartiteMatchingAlgorithm} that avoid the overhead of setting up a weighted digraph suitable for network flow.
%%%%%
%%%%%
\end{document}
